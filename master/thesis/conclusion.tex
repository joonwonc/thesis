\chapter{Discussion and Conclusions}
\label{chap:conclusions}

\paragraph{Inlining for Parametrized Modules}

The inlining operator $\inlineF{\cdot}$ manipulates module definitions
in a static way. Therefore, it requires a computation with module
structures. All definitions about the operator generally perform two
tasks; they either change action structures (by substituting call
sites to their actual bodies), or compare two method names (in order
to check a method call name equals the inlining target while examining
actions). Each definition is either an ordinary function or a
recursive function -- there are no relational definitions. We call
such definitions \emph{computational}, which informally means that no
proofs are involved.

However, in the Coq proof assistant on which \Kami{} is built,
variables make the computation stuck. \Kami{} allows module
definitions to be \emph{parametric}, aiming for general designs and
corresponding verifications. For example, a fifo implementation can be
parametrized with respect to the size of the fifo. For a processor
implementation, various sizes such as addresses, values, and register
files also can be parametrized.  However, once a module has
parameters, inlining cannot be computationally performed, since
variables cannot be compared unless they have concrete values. As a
result, the inlining operation is no longer computational when it
comes to parametrzied modules.

In order to solve this problem, a \emph{term rewriting theorem} was
defined (proved) and has been used to rewrite parameter
comparisons. Even if we cannot define a general way to compare two
parameters without any hypotheses, at least we can prove a reflexive
term rewriting theorem, saying $\forall s.\; \textsf{is\_equal}\ s\ s
= \btrue$. It is certainly not a complete theorem to include all
possible parameter-comparison scenarios. However, we claim this kind
of theorem is enough to perform inlining for practical designs. For
instance, parameters are used to index processors for designing
multi-core processors. However, since processors do not communicate
directly with each others, method calls are used only within
processors. This design fact indicates that parameter comparisons
always show two aspects: either they have different concrete prefixes
($p \neq q \to p_i \neq q_j$), or they are exactly the same even with
the index ($p_i = p_i$). Therefore, in this case, the term rewriting
for reflexivity is enough to rewrite comparisons.

However, because of the term rewriting, the inlining operator loses
its original performance. Term rewriting is considered to be slow in
Coq. For a simple processor definition (approximately $200$ lines of
code), it took around $30$ seconds to inline the processor module. One
might say this is tolerable, but considering inlining as a trivial and
simple operation, it may not be tolerable. Furthermore, a profiler in
Coq produced a result that over $80$ percentage of the time was due to
term rewritings; inlining could have taken around $5$ seconds without
parameters.

\paragraph{Inlining and Synthesis}

Inlining is designed for efficient proofs, not for synthesis. Inlined
modules can be synthesized into actual hardware components, but they
should not be used as synthesis targets. The main reason is
performance. In terms of synthesis, substituting action bodies to call
sites implies that the corresponding circuit is duplicated throughout
the module. It certainly increases the size of the synthesized
circuit.

Thanks to the semantic implication proof, we can synthesize an
original module, while verification is performed on the inlined
one. Suppose that we want to prove a property about a module $m$,
saying $P(m)$, where $P$ is related to semantics. The semantic
implication theorem (\refthm{thm-modtoinl} in
\refchap{chap:implication}) allows us to reduce the goal to
$P'(\inlineF{m})$, where $P'$ is also related to semantics. $P'$ is
generally easier to prove, since we can take advantage of properties
of inlined modules; \eg{} there are no internal calls in inlined
modules.

\paragraph{Conclusions and Future Work}

We have built the \Kami{} framework for general hardware verification,
which requires a series of general concepts. With the belief that
modularity is the key concept of hardware designs, we have defined
modular semantics. However, modular semantics has an inherent weakness
in that it is hard to infer internal state changes. Hence, we adopt a
new semantic concept based on inlining in order to eliminate the
weakness. We also prove an implication from the modular semantics to
the inlined one, thus the inlining can be freely employed during
verification.

The grand goal of the \Kami{} framework is to provide efficient tools
for general and practical hardware verification. Modular semantics and
inlining are the main tools of the framework. In addition to them, we
have built other interfaces such as renaming, decomposition, and
modular refinement~\cite{murali-thesis}.

Using these various tools, we aim to verify real-world hardware
components. Modern hardware designs such as a pipelined processor or a
multi-level cached memory are too complicated to provide completely
general verifications. However, we believe that efficient and
high-level verification interfaces can make such verifications much
easier. Modular semantics and the refinement theorem will separate
verification into small pieces. Inlining will help us easily verify
each small component. Equipped with an effective framework, it is no
longer a dream to verify real-world hardware.

