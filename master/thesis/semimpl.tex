\chapter{Proving the Implication from Modular to Inlining Semantics}
\label{chap:implication}

\section{Meaning of the Semantic Implication}

The semantics defined so far does not have equal capabilities. The
inlining semantics can simulate more design cases than the modular
semantics. In other words, it is possible to prove an implication from
the modular semantics to the inlining one, but not for the inverse.

\begin{figure}[t]
  \begin{subfigure}[b]{0.5\textwidth}
    \bsvmodnoreg{m}{
      \bsvnone{\pgmmeth}{f}{}{
        \pgmwrite{r_1}{1}
        \pgmcall{x}{h}{}
        \pgmwrite{r_3}{x}
      }\\
      \bsvnone{\pgmmeth}{g}{}{
        \pgmwrite{r_2}{2}
        \pgmcall{x}{h}{}
        \pgmwrite{r_4}{x}
      }\\
      \bsvnone{\pgmmeth}{h}{}{
        \pgmrets{} 0
      }
    }
    \subcaption{A module before inlining}
  \end{subfigure}
  \begin{subfigure}[b]{0.5\textwidth}
    \bsvmodnoreg{\inlineF{m}}{
      \bsvnone{\pgmmeth}{f}{}{
        \pgmwrite{r_1}{1}
        \pgmletin{x}{0}
        \pgmwrite{r_3}{x}
      }\\
      \bsvnone{\pgmmeth}{g}{}{
        \pgmwrite{r_2}{2}
        \pgmletin{x}{0}
        \pgmwrite{r_4}{x}
      }
    }
    \subcaption{A module after inlining}
  \end{subfigure}
  \caption{Inlining semantics allows an execution of $f$ and $g$,
    while the modular one does not.}
  \label{ex-inlining-covers-more}
\end{figure}

\reffig{ex-inlining-covers-more} shows the case that the inlining
semantics allows a specific execution, while the modular one does
not. In a module $m$, methods $f$ and $g$ cannot be executed
concurrently, since both methods call a method $h$. More specifically,
the \Substeps{} judgement containing substeps for $f$ and $g$ cannot
be constructed due to the disjointness condition of called
methods. However, in the inlined module $\inlineF{m}$, $f$ and $g$ can
be executed concurrently, since $h$ simply returns a constant zero.

Does this case hurt the capacity of the modular semantics? According
to the example, when $h$ does not write registers or call methods, any
two methods both calling $h$ should be able to be executed
concurrently. This case is quite practical in real-world hardware
design; for example, if $m$ has a register $r_{\textrm{cnt}}$ which
acts as a counter, and $h$ calculates the next value of the counter by
returning $(r_{\textrm{cnt}} + 1)$, any rules and methods should be
able to call $h$, even though they are executed concurrently.

We can resolve this issue by converting designs to the ones without
such methods by inlining. Bluespec has two notations for methods,
called $\textbf{ActionMethod}$ and
$\textbf{Method}$. $\textbf{ActionMethod}$ can write registers or call
methods, while $\textbf{Method}$ cannot contain such actions. Hence,
if we inline all $\textbf{Method}$s while converting designs from
\Bluespec{} to \Kami{}, the issue will not happen (though the inlining
in conversion will remain as a part of trusted base).

Based on the analysis so far, we give the main theorem for semantic
implication. \refthm{thm-modtoinl} claims that the set of possible
behaviors of the modular semantics is a subset of the one in the
inlining semantics.

\begin{theorem}[Modular semantics imply inlining semantics]
  \label{thm-modtoinl}
  \mbox{}
  \begin{center}
    \begin{math}
      \forall m.\forall o.\forall u.\forall l.\ 
      (\semstep{m}{o}{u}{l}) \Longrightarrow (\semstepin{m}{o}{u}{l})
    \end{math}
  \end{center}
\end{theorem}

\section{The Implication Proof: From Modular to Inlining Semantics}

On this section, we present the proof of \refthm{thm-modtoinl}. The
proof inductively follows the definition of the complete inlining
operator $\inlineF{\cdot}$. In other words, we gradually present the
correctness lemma from the concatenation operator $(\concatsymb)$,
which is the most basic operator for inlining, to the complete
inlining operator.

All correctness lemmas have a similar form, which describes that if we
have a judgement for a target object which calls a target method and a
judgement for the target method, then two judgements can be
\emph{merged} by a proper inlining operator. More formally,
correctness lemmas have a following form:
\begin{center}
  \begin{math}
    \begin{array}{l}
      (\textrm{a judgement calling an inlining method}) \to \\
      (\textrm{conditions}) \to \\
      (\textrm{a judgement of an inlining method}) \to \\
      (\textrm{a merged judgement where the method is inlined by a certain operator}).
    \end{array}
  \end{math}
\end{center}

\paragraph{Correctness of the concatenation operator}

Since the concatenation operator $(\concatsymb)$ merely appends two
actions, the correctness lemma also states the corresponding semantic
merge.

\begin{lemma}[Correctness of the action concatenation]
  \label{lem-concatsymb}
  \mbox{}\\
  For every $o, a_1, u_1, cs_1, v_1, a_2, u_2, cs_2,$ and $v_2,$
  \begin{center}
    \begin{math}
      \begin{array}{l}
        \semact{o}{a_1}{u_1}{cs_1}{v_1} \to \\
        \semact{o}{(\lambda x.a_2)\ v_1}{u_2}{cs_2}{v_2} \to \\
        \semact{o}{\concataction{a_1}{a_2}}{\sunion{u_1}{u_2}}
               {\lunion{cs_1}{cs_2}}{v_2}
      \end{array}
    \end{math}
  \end{center}
\end{lemma}
\begin{proof}
  Straightforward by the inductive definition of $(\concatsymb)$.
\end{proof}

Note that two updates $u_1$ and $u_2$ and two called methods $cs_1$
and $cs_2$ \emph{do not need to be disjoint} each other -- a normal
union $\cup$ is used to merge updates and method calls, instead of a
disjoint union $\uplus$. This is because the action semantics allows
double writes or calls. The conditions are defined as one of the
well-formedness conditions, which is independent to the semantics.

\paragraph{Correctness of the method-inlining operator}

Using the correctness lemma for the concatenation operator, we give
one for the method-inlining operator. The correctness lemma is
described first with action structures, and lifted to \Substep{} and
\Substeps{}.

There are two variations of the correctness lemma: the \emph{intact
  case} is designed for objects which are not related to the method
which is inlined. For example, if an action does not call the method,
then a judgement of the action should not be changed. The \emph{call
  case} is designed for objects which call the method. In this case,
we will get a merged judgement as explained.

\begin{lemma}[Correctness of $(\inlinedmsymb)$ for actions -- intact case]
  \label{lem-inlinedm-action-intact}
  \mbox{}\\
  For every $o, a_f, a, u, cs, v_r,$ and $f,$
  \begin{center}
    \begin{math}
      \begin{array}{l}
        \semact{o}{a}{u}{cs}{v_r} \to \\
        f \notin cs \to \\
        \semact{o}{\inlinedm{a}{(f, a_f)}}{u}{cs}{v_r}
      \end{array}
    \end{math}
  \end{center}
\end{lemma}
\begin{proof}
  By induction on the action judgement for $a$. The only nontrivial
  case is a method call action. If $a = (\actcall{x}{f'}{e}{a'})$,
  then $f'$ cannot be equal to $f$ since $f' \in cs$ by the inductive
  definition of the method call, but $f \notin cs$. Therefore,
  $\langle u, cs, v_r \rangle$ is never changed by the inlining.
\end{proof}

Below call case is where the well-formedness condition occurs firstly.

\begin{lemma}[Correctness of $(\inlinedmsymb)$ for actions -- call case]
  \label{lem-inlinedm-action-call}
  \mbox{}\\
  For every $o, a_f, a, u_f, u, cs_f, cs, v_r, f, v_a, v_r,$ and $v_f,$
  \begin{center}
    \begin{math}
      \begin{array}{l}
        \semact{o}{a}{u}{cs}{v_r} \to \\
        \wfdoublea{a} \to \\
        \sdisj{u_f}{u} \to \mdisj{cs_f}{cs} \to \\
        cs[f] = (v_a, v_f) \to \\
        \semact{o}{(\lambda x.a_f)\ v_a}{u_f}{cs_f}{v_f} \to \\
        \semact{o}{\inlinedm{a}{(f, a_f)}}{\splus{u_f}{u}}{\mplus{cs_f}{cs}}{v_r}
      \end{array}
    \end{math}
  \end{center}
\end{lemma}
\begin{proof}
  By induction on the action judgement for $a$. The only nontrivial
  case is a method call action. When $a = (\actcall{x}{f'}{e}{a'})$,
  we have to consider following two cases:
  \begin{itemize}
  \item If $f' = f$, then the method call should have the argument
    value $v_a$ and the return value $v_r$, since $cs[f] = (v_a,
    v_f)$. Since $(\inlinedmsymb)$ is defined using $(\concatsymb)$,
    we can use the judgement for $(\lambda x.a_f)\ v_a$ and
    \reflem{lem-concatsymb} to prove the goal.
  \item If $f' \neq f$, then it is same as the intact case.
  \end{itemize}
\end{proof}

When is the well-formedness condition $(\wfdouble{a})$ used during the
proof? Suppose there is a double call in $a$. If the argument-return
pairs of the two calls differ, then one of them is overwritten by the
other one while evaluating $cs$, according to the action semantics. In
this case, when encountering the method call where the pair is
overwritten, we cannot apply \reflem{lem-concatsymb} since the pair
does not match.

\paragraph{Correctness of the method-inlining operator for modules}

We lift \reflem{lem-inlinedm-action-intact} (the intact case) and
\reflem{lem-inlinedm-action-call} (the call case) to the level of
\Substep{} and \Substeps{}.

\begin{lemma}[Correctness of $(\inlinedmmsymb)$ for \Substep{} -- intact case]
  \label{lem-inlinedmm-substep-intact}
  \mbox{}\\
  For every $m, o, u, ds, cs,$ and $f,$
  \begin{center}
    \begin{math}
      \begin{array}{l}
        \semsstep{m}{o}{u}{\alpha}{ds}{cs} \to \\
        f \notin ds \to \\
        \semsstep{\inlinedmm{m}{f}}{o}{u}{\alpha}{ds}{cs}
      \end{array}
    \end{math}
  \end{center}
\end{lemma}
\begin{proof}
  Straightforward by the definition of \Substep{}. Destructing
  \Substep{} gives two cases: one for a rule, and the other for a
  method. In both cases, we directly apply
  \reflem{lem-inlinedm-action-intact} to prove the goal.
\end{proof}

\begin{lemma}[Correctness of $(\inlinedmmsymb)$ for \Substep{} -- call case]
  \label{lem-inlinedmm-substep-call}
  \mbox{}\\
  For every $m, o, u_f, u, ds, cs_f, cs, f, a_f, v_a,$ and $v_r,$
  \begin{center}
    \begin{math}
      \begin{array}{l}
        \semsstep{m}{o}{u}{\alpha}{ds}{cs} \to \\
        \wfdouble{m} \to \\
        \semact{o}{(\lambda x.a_f)\ v_a}{u_f}{cs_f}{v_r} \to \\
        \sdisj{u_f}{u} \to \mdisj{cs_f}{cs} \to \\
        cs[f] = (v_a, v_r) \to \\
        \semsstep{\inlinedmm{m}{f}}{o}{\splus{u_s}{u}}
                 {\alpha}{ds}{\mapfilt{\mplus{cs_s}{cs}}{f}}
      \end{array}
    \end{math}
  \end{center}
\end{lemma}
\begin{proof}
  Straightforward by the definition of \Substep{}. Destructing
  \Substep{} gives two cases: one for a rule, and the other for a
  method. In both cases, we directly apply
  \reflem{lem-inlinedm-action-call} to prove the goal.
\end{proof}

Note that $(\wfdouble{m})$ is given as a condition instead of
$(\wfdoublea{a})$. This is because we cannot directly get the action
body from \Substep{}. Since $(\wfdouble{m})$ implies that each action
$a$ in $m$ satisfies $(\wfdoublea{a})$, we eventually get what is
needed.

Now we lift two theorems \reflem{lem-inlinedmm-substep-intact} and
\reflem{lem-inlinedmm-substep-call} to the level of \Substeps{}. Since
\Substeps{} is composed of a series of \Substep{}, we have more
fine-grained cases in order to describe possible cases of each
\Substep{}:
\begin{itemize}
\item An \emph{intact case} (\reflem{lem-inlinedmm-intact}) describes
  the case where the inlined method is not called anywhere in the
  substeps. As shown in above similar lemmas, the judgement will not
  be changed.
\item A \emph{rule case} (\reflem{lem-inlinedmm-rule}) describes the
  case where a substep in the substeps is defined by a rule, and the
  rule calls the inlined method. Also, the inlined method is defined
  as a substep in the substeps.
\item A \emph{method case 1} (\reflem{lem-inlinedmm-meth1}) describes
  the case where a substep in the substeps is defined by a method $g$,
  but it is not the inlined method. Also, the substeps call $g$
  somewhere.
\item A \emph{method case 2} (\reflem{lem-inlinedmm-meth2}) describes
  the case where a substep in the substeps is defined by a method $f$,
  which is the inlined method.
\end{itemize}

\begin{lemma}[Correctness of $(\inlinedmmsymb)$ for \Substeps{} -- intact case]
  \label{lem-inlinedmm-intact}
  \mbox{}\\
  For every $m, o, u, l,$ and $f,$
  \begin{center}
    \begin{math}
      \begin{array}{l}
        \semsss{m}{o}{u}{l} \to \\
        f \notin (\callsof{l}) \to \\
        \semsss{\inlinedmm{m}{f}}{o}{u}{l}
      \end{array}
    \end{math}
  \end{center}
\end{lemma}
\begin{proof}
  By induction on \Substeps{}. Each substep in the substeps does not
  call $f$, since $f \notin (\callsof{l})$ and $l$ is the merged label
  of all substeps, according to the definition. Thus, by
  \reflem{lem-inlinedmm-substep-intact}, for each substep
  \semsstepr{m}{o}{u_i}{l_i}, we get
  \semsstepr{\inlinedmm{m}{f}}{o}{u_i}{l_i}. Since $l = \uplus_i\ l_i$,
  we get the goal by the definition of \Substeps{}.
\end{proof}

\begin{lemma}[Correctness of $(\inlinedmmsymb)$ for \Substeps{} -- rule case]
  \label{lem-inlinedmm-rule}
  \mbox{}\\
  For every $m, o, u_s, u, ds, cs_s, cs, s,$ and $f,$
  \begin{center}
    \begin{math}
      \begin{array}{l}
        \semsstep{m}{o}{u_s}{\alpharule{s}}{\emptymap}{cs_s} \to \\
        \wfdouble{m} \to \\
        \semsss{m}{o}{u}{\semlbl{\alphameth}{ds}{cs}} \to \\
        f \in cs_s \to f \in ds \to \sdisj{u_s}{u} \to \mdisj{cs_s}{cs} \to \\
        \semsss{\inlinedmm{m}{f}}{o}{\splus{u_s}{u}}{\semlbl{\alpharule{s}}
          {\mapfilt{ds}{f}}{\mapfilt{\mplus{cs_s}{cs}}{f}}}
      \end{array}
    \end{math}
  \end{center}
\end{lemma}
\begin{proof}
  By induction on substeps containing $f \in ds$. By the definition of
  \Substeps{}, a substep for $f$ exists in the substeps. Using the
  substep for $f$ and the substep for the rule $s$ calling $f$, we get
  the merged substep by \reflem{lem-inlinedmm-substep-call}. The other
  substeps except for $f$ are treated as an intact case, since we get
  $f \notin cs$ by $f \in cs_s$ and $\mdisj{cs_s}{cs}$.
\end{proof}

\begin{lemma}[Correctness of $(\inlinedmmsymb)$ for \Substeps{} -- method case 1]
  \label{lem-inlinedmm-meth1}
  \mbox{}\\
  For every $m, o, u_f, u, \alpha, ds, cs_f, cs, g, v_a, v_r,$ and $f,$
  \begin{center}
    \begin{math}
      \begin{array}{l}
        \semsstep{m}{o}{u_f}{\alphameth}{\stupd{}{g}{(v_a, v_r)}}{cs_f} \to \\
        \wfdouble{m} \to \\
        \semsss{m}{o}{u}{\semlbl{\alpha}{ds}{cs}} \to \\
        f \in cs_f \to f \in ds \to g \notin ds \to \\
        \sdisj{u_f}{u} \to \mdisj{cs_f}{cs} \to \\
        \semsss{\inlinedmm{m}{f}}{o}{\splus{u_f}{u}}{\semlbl{\alpha}
          {\mapfilt{\mplus{\stupd{}{g}{(v_a, v_r)}}{ds}}{f}}
          {\mapfilt{\mplus{cs_f}{cs}}{f}}}
      \end{array}
    \end{math}
  \end{center}
\end{lemma}
\begin{proof}
  The proof is similar to \reflem{lem-inlinedmm-rule}, done by
  induction on substeps containing $f \in ds$. A substep for $f$
  exists in the substeps, and with the substep for $g$, we get the
  merged substep by \reflem{lem-inlinedmm-substep-call}. The other
  substeps except for $f$ are treated as an intact case. Additionally,
  we do not have a conflict on defined methods while merging, since $g
  \notin ds$.
\end{proof}

\begin{lemma}[Correctness of $(\inlinedmmsymb)$ for \Substeps{} -- method case 2]
  \label{lem-inlinedmm-meth2}
  \mbox{}\\
  For every $m, o, u_f, u, \alpha, ds, cs_f, cs, v_a, v_r,$ and $f,$
  \begin{center}
    \begin{math}
      \begin{array}{l}
        \semsstep{m}{o}{u_f}{\alphameth}{\stupd{}{f}{(v_a, v_r)}}{cs_f} \to \\
        \wfdouble{m} \to \\
        \semsss{m}{o}{u}{\semlbl{\alpha}{ds}{cs}} \to \\
        f \in cs \to f \notin ds \to \\
        \sdisj{u_f}{u} \to \mdisj{cs_f}{cs} \to \\
        \semsss{\inlinedmm{m}{f}}{o}{\splus{u_f}{u}}{\semlbl{\alpha}{ds}
          {\mapfilt{\mplus{cs_f}{cs}}{f}}}
      \end{array}
    \end{math}
  \end{center}
\end{lemma}
\begin{proof}
  The proof is similar to \reflem{lem-inlinedmm-meth1}, done by
  induction on substeps calling $f \in cs_f$. A substep calling $f$
  exists in the substeps, and with the substep for $f$, we get the
  merged substep by \reflem{lem-inlinedmm-substep-call}. The other
  substeps cannot call $f$ by the definition of \Substeps{}, thus they
  are treated as an intact case.
\end{proof}

\paragraph{Giving the relation between a label and inlining}

Before proving the correctness of the method-inlining operator
$(\inlinedmmsymb)$, which can be proved by applying the four lemmas
(\reflem{lem-inlinedmm-intact}, \reflem{lem-inlinedmm-rule},
\reflem{lem-inlinedmm-meth1}, and \reflem{lem-inlinedmm-meth2}) for
each proper case, we explain an intuitive relation between a label and
inlining. The intuition starts from a basic question: what is the
meaning of inlining in terms of semantics?

\begin{figure}[t]
  \begin{subfigure}[b]{0.5\textwidth}
    \bsvmodnoreg{m}{
      \bsvnone{\pgmmeth}{f}{}{
        \pgmwrite{r_1}{1}
        \pgmcalln{g}{}
      }\\
      \bsvnone{\pgmmeth}{g}{}{
        \pgmwrite{r_2}{2}
        \pgmcalln{h}{}
      }\\
      \bsvnone{\pgmmeth}{h}{}{
        \pgmwrite{r_3}{3}
      }
    }
    \subcaption{Before inlining $g$}
  \end{subfigure}
  \begin{subfigure}[b]{0.5\textwidth}
    \bsvmodnoreg{\inlinedmm{m}{g}}{
      \bsvnone{\pgmmeth}{f}{}{
        \pgmwrite{r_1}{1}
        \pgmwrite{r_2}{2}
        \pgmcalln{h}{}
      }\\
      \bsvnone{\pgmmeth}{h}{}{
        \pgmwrite{r_3}{3}
      }
    }
    \subcaption{After inlining $g$}
  \end{subfigure}
  \caption{Inlining a method hides label elements of the method}
  \label{ex-inlining-hide}
\end{figure}

When inlining a method, label elements related to the method are
hidden. \reffig{ex-inlining-hide} presents an example where a method
$g$ is inlined. Considering a step judgements of $f$, before inlining,
corresponding substeps should contain $f, g,$ and $h$ since $f$ calls
$g$ and $g$ calls $h$. Thus, the label of the substeps should contain
$f, g,$ and $h$. Whereas, after inlining, corresponding substeps now
contain $f$ and $h$, thus the label contains only $f$ and
$h$. Therefore, inlining a method is interpreted as \emph{hiding the
  method from a label}, in terms of semantics.

Following the intuition, we define \hidemethsym{} and \hidemethssym{}
to form labels in which specific methods are
hidden. $(\hidemeth{l}{f})$ hides a method $f$ from a given label $l$,
which is defined as follows:
\begin{definition}
  \label{def-hidemeth}
  \mbox{}
  \begin{center}
    \begin{math}
      \begin{array}{lcl}
        \hidemeth{l}{f} & = & \wordif{} (\defsof{l})[f] = (\callsof{l})[f] \\
        & & \wordthen{} \semlbl{\annotof{l}}{\mapfilt{\defsof{l}}{f}}{\mapfilt{\callsof{l}}{f}} \\
        & & \wordelse{} l
      \end{array}
    \end{math}
  \end{center}
\end{definition}
$(\hidemeths{l}{fs})$ is a natural extension of \hidemethsym{}, hiding
list of methods $fs$ from a given label $l$:
\begin{definition}
  \label{def-hidemeths}
  \mbox{}
  \begin{center}
    \begin{math}
      \begin{array}{lcl}
        \hidemeths{l}{\emptymap} & = & l \\
        \hidemeths{l}{(f : fs)} & = & \hidemeths{(\hidemeth{l}{f})}{fs}
      \end{array}
    \end{math}
  \end{center}
\end{definition}

Finally, we are ready to state the lemma for the correctness of
$(\inlinedmmsymb)$. For given substeps, they either call the inlining
target method or not. If they call the method, they are required to
have a substep for the method in order to use the substep to form the
merged substep.

\begin{lemma}[Correctness of $(\inlinedmmsymb)$]
  \label{lem-inlinedmm}
  \mbox{}\\
  For every $m, o, u, l,$ and $f,$
  \begin{center}
    \begin{math}
      \begin{array}{l}
        \semsss{m}{o}{u}{l} \to \\
        f \notin (\callsof{l}) \vee (\defsof{l})[f] = (\callsof{l})[f] \to \\
        \semsss{\inlinedmm{m}{f}}{o}{u}{\hidemeth{l}{f}}
      \end{array}
    \end{math}
  \end{center}
\end{lemma}
\begin{proof}
  The proof is largely divided into two cases: $f \notin
  (\callsof{l})$ or $(\defsof{l})[f] = (\callsof{l})[f]$.
  \begin{itemize}
  \item If $f \notin (\callsof{l})$, it is exactly an intact case so
    we apply \reflem{lem-inlinedmm-intact} to get the goal. Note that
    we get $(\hidemeth{l}{f}) = l$, because $f \notin (\callsof{l})$.
  \item If $(\defsof{l})[f] = (\callsof{l})[f]$, the proof is done by
    induction on the given substeps. Picking a head substep from the
    substeps forms following four cases:
    \begin{itemize}
    \item If the head substep is defined for $f$, then
      \reflem{lem-inlinedmm-meth2} is applied to get the goal. $cs_f$
      in the lemma cannot contain $f$, since the substep is defined
      for $f$ and we know $(\neg\; \isrec{f})$ ($f$ is not
      self-recursive), which is brought from the definition of
      $(\inlinedmmsymb)$.  Thus, we get $f \in cs$ so
      \reflem{lem-inlinedmm-meth2} can be applied properly.
    \item If the head substep is defined for a method $g$ where $g
      \neq f$, but calls $f$, then \reflem{lem-inlinedmm-meth1} is
      applied to get the goal. All conditions of the lemma are
      introduced by the definition of \Substeps{}.
    \item If the head substep is defined for a rule $s$, but calls
      $f$, then \reflem{lem-inlinedmm-rule} is applied to get the
      goal. Similarly, all conditions are introduced by the definition
      of \Substeps{}.
    \item If the head substep is defined not for $f$ and does not call
      $f$, then \reflem{lem-inlinedmm-substep-intact} is applied to
      get the substep after inlining. The induction hypothesis gives
      the substeps after inlining. Thus we simply merge the substep
      and the substeps to get the goal.
    \end{itemize}
  \end{itemize}
\end{proof}

\paragraph{Correctness of the methods-inlining operator}

\reflem{lem-inlinedmssubo} is simply an extension of the correctness
of the method-inlining operator, described in \reflem{lem-inlinedmm}.

\begin{lemma}[Correctness of $(\inlinedmssymb)$ w.r.t. \hidemethssym{}]
  \label{lem-inlinedmssubo}
  \mbox{}\\
  For every $m, o, u, l,$ and $fs,$
  \begin{center}
    \begin{math}
      \begin{array}{l}
        \semsss{m}{o}{u}{l} \to \\
        \wellhidden{m}{(\hide{l})} \to \\
        \semsss{\inlinedms{m}{fs}}{o}{u}{\hidemeths{l}{fs}}
      \end{array}
    \end{math}
  \end{center}
\end{lemma}
\begin{proof}
  Straightforward by induction on $fs$. For every element $f$ of $fs$,
  we can apply \reflem{lem-inlinedmm} to get the goal. Note that the
  necessary condition $(f \notin (\callsof{l}) \vee (\defsof{l})[f] =
  (\callsof{l})[f])$ always holds by the condition
  $(\wellhidden{m}{(\hide{l})})$.
\end{proof}

\reflem{lem-hidemeths} fills the gap between \hidemethssym{} and
\hidesym{}, which implies that when inlining is done for all defined
methods in a module, then its effect to the label is as same as the
\hidesym{} operation.

\begin{lemma}[Relation between \hidemethssym{} and \hidesym{}]
  \label{lem-hidemeths}
  \mbox{}\\
  For every $l$ and $fs$,
  \begin{center}
    \begin{math}
      \begin{array}{l}
        \defsof{l} \subseteq fs \to \hidemeths{l}{fs} = \hide{l}
      \end{array}
    \end{math}
  \end{center}
\end{lemma}
\begin{proof}
  Straightforward by the definitions of \hidemethssym{} and
  \hidesym{}.
\end{proof}

Finally, by using \reflem{lem-inlinedmssubo} and
\reflem{lem-hidemeths}, we can prove the correctness of
$(\inlinedmssymb)$.

\begin{lemma}[Correctness of $(\inlinedmssymb)$ w.r.t. \Substeps{}]
  \label{lem-inlinedmssub}
  \mbox{}\\
  For every $m, o, u, l,$ and $fs,$
  \begin{center}
    \begin{math}
      \begin{array}{l}
        \semsss{m}{o}{u}{l} \to \\
        \wellhidden{m}{(\hide{l})} \to \\
        \semsss{\inlinedms{m}{(\namesof{(\methsofm{m})})}}{o}{u}{\hide{l}}
      \end{array}
    \end{math}
  \end{center}
\end{lemma}
\begin{proof}
  Straightforward by applying \reflem{lem-inlinedmssubo} and
  \reflem{lem-hidemeths}. A statement $\defsof{l} \subseteq
  (\namesof{(\methsofm{m})})$ is trivial, since the label is always
  formed by merging substeps constructed by a rule or defined methods.
\end{proof}

\begin{lemma}[Correctness of $(\inlinedmssymb)$]
  \label{lem-inlinedms}
  \mbox{}\\
  For every $m, o, u, l,$ and $fs,$
  \begin{center}
    \begin{math}
      \begin{array}{l}
        (\semstep{m}{o}{u}{l}) \to \\
        fs = \namesof{(\methsofm{m})} \to \\
        (\semstep{(\inlinedms{m}{fs})}{o}{u}{l})
      \end{array}
    \end{math}
  \end{center}
\end{lemma}
\begin{proof}
  This lemma is almost as same as \reflem{lem-inlinedmssub}. The only
  gap is filled by showing $\hide{(\hide{l})} = \hide{l}$, which can
  be proved easily by destructing the definition of \hidesym{}.
\end{proof}

\paragraph{Correctness of the inlining operator}

At last, we show the correctness of the overall inlining
operation. Since \reflem{lem-inlinedms} directly gives the correctness
of $\inline{\cdot}$, it suffices to consider the internal method
filtering defined in $\inlineF{\cdot}$.

\begin{lemma}[Correctness of $\inline{\cdot}$]
  \label{lem-inline}
  For every $m, o, u,$ and $l,$
  \begin{center}
    \begin{math}
      \begin{array}{l}
        (\semstep{m}{o}{u}{l}) \to (\semstep{\inline{m}}{o}{u}{l})
      \end{array}
    \end{math}
  \end{center}
\end{lemma}
\begin{proof}
  By the definition of $\inline{\cdot}$, it is simply another
  representation of \reflem{lem-inlinedms}.
\end{proof}

\begin{lemma}[Correctness of $\inlineF{\cdot}$]
  \label{lem-inlinef}
  For every $m, o, u,$ and $l,$
  \begin{center}
    \begin{math}
      \begin{array}{l}
        (\semstep{m}{o}{u}{l}) \to (\semstep{\inlineF{m}}{o}{u}{l})
      \end{array}
    \end{math}
  \end{center}
\end{lemma}
\begin{proof}
  Applying \reflem{lem-inline}, if suffices to show:
  \begin{math}
    (\semstep{\inline{m}}{o}{u}{l}) \to
    (\semstep{\inlineF{m}}{o}{u}{l}).
  \end{math}
  According to the definition $\inlineF{m} \triangleq
  (\regsofm{\inline{m}}, \rulesofm{\inline{m}},
  \mapfilt{(\methsofm{\inline{m}})}{(\callsofm{m})})$, the rules are
  preserved and the methods are filtered whether they are called
  internally or not. Now for every substep in the step:
  \begin{itemize}
  \item If it is defined by a rule, the substep is trivially
    preserved.
  \item If if is defined by a method, it suffices to show that the
    label $l$ does not contain anything about internal
    calls. $(\defsof{l})$ cannot contain internal calls, since
    $\wellhidden{m}{l}$ by the definition of \Step{}. $(\callsof{l})$
    cannot contain internal calls, since $\inline{m}$ does not have
    internal calls any longer by the definition. Thus, $l$ is
    independent to the internal calls, which implies that the substep
    is also preserved.
  \end{itemize}
\end{proof}

Note that now \refthm{thm-modtoinl} is no more than another
representation of \reflem{lem-inlinef}, according to the definition of
the inlining semantics.

