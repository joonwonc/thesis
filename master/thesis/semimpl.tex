\chapter{Proving Implications Between Semantics}
\label{chap-implications}

\section{Semantic Implications} %$ which semantics are more powerful?

The semantics defined so far does not have equal capabilities. The
inlining semantics can deal with more design cases than the modular
semantics. The big-step semantics also cover more cases than the
modular semantics. In other words, we can prove implications from the
modular semantics to the inlining or the big-step ones, but cannot
prove inverses.

\begin{figure}[t]
  \begin{subfigure}[b]{0.5\textwidth}
    \bsvmodnoreg{m}{
      \bsvnone{\pgmmeth}{f}{}{
        \pgmwrite{r_1}{1}
        \pgmcall{x}{h}{}
        \pgmwrite{r_3}{x}
      }\\
      \bsvnone{\pgmmeth}{g}{}{
        \pgmwrite{r_2}{2}
        \pgmcall{x}{h}{}
        \pgmwrite{r_4}{x}
      }\\
      \bsvnone{\pgmmeth}{h}{}{
        \pgmrets{} 0
      }
    }
    \subcaption{A module before inlining}
  \end{subfigure}
  \begin{subfigure}[b]{0.5\textwidth}
    \bsvmodnoreg{\inlineF{m}}{
      \bsvnone{\pgmmeth}{f}{}{
        \pgmwrite{r_1}{1}
        \pgmletin{x}{0}
        \pgmwrite{r_3}{x}
      }\\
      \bsvnone{\pgmmeth}{g}{}{
        \pgmwrite{r_2}{2}
        \pgmletin{x}{0}
        \pgmwrite{r_4}{x}
      }
    }
    \subcaption{A module after inlining}
  \end{subfigure}
  \caption{Inlining semantics allow an execution of $f$ and $g$, while
    the modular one does not allow.}
  \label{ex-inlining-covers-more}
\end{figure}

\reffig{ex-inlining-covers-more} shows the case where the inlining
semantics allow a specific execution, while the modular one does not
allow. In a module $m$, methods $f$ and $g$ cannot be executed
concurrently, since both methods call a method $h$. More specifically,
\Substeps{} containing the substeps for $f$ and $g$ cannot be
constructed due to the disjointness condition of called
methods. However, in the inlined module $\inlineF{m}$, $f$ and $g$ can
be executed concurrently, since $h$ simply returns a constant zero.

Does this case hurt the capacity of the modular semantics? According
to the example, when $h$ does not write registers or call methods, any
two methods both calling $h$ should be able to be executed
concurrently. This case is quite practical in hardware design; for
example, if $m$ has a register $r_{\textrm{cnt}}$ which acts as a
counter, and $h$ calculates the next value of the counter by returning
$(r_{\textrm{cnt}} + 1)$, any rules and methods should be able to call
$h$, even though they are executed concurrently.

We can resolve this issue by converting programs to the ones without
such methods by inlining. Bluespec has two notations for methods,
called $\textbf{ActionMethod}$ and
$\textbf{Method}$. $\textbf{ActionMethod}$ can write registers or call
methods, while $\textbf{Method}$ cannot contain such actions. Hence,
if we inline all $\textbf{Method}$s while converting from \Bluespec{}
to \Kami{}, the issue will not happen (though the inlining in
conversion will remain as a part of trusted base).

The big-step semantics deal with more design cases than the modular
one, by the same reason. According to the big-step semantics, when we
encounter $h$, it is not added to the called methods, since it is an
internal call (see \refdef{def-sembigact} for details). Hence, a
double call does not occur when we execute $f$ and $g$ in a big-step
manner.

From the analysis so far, we give two main theorem for semantic
implications. \refthm{thm-modtoinl} claims that the set of possible
behaviors of the modular semantics is a subset of the one in the
inlining semantics. \refthm{thm-modtobig} claims the similar one with
respect to the big-step semantics.

\begin{theorem}[Modular semantics imply inlining semantics]
  \label{thm-modtoinl}
  \mbox{}
  \begin{center}
    \begin{math}
      \forall m.\forall o.\forall u.\forall l.\ 
      (\semstep{m}{o}{u}{l}) \Longrightarrow (\semstepin{m}{o}{u}{l})
    \end{math}
  \end{center}
\end{theorem}

\begin{theorem}[Modular semantics imply big-step semantics]
  \label{thm-modtobig}
  \mbox{}
  \begin{center}
    \begin{math}
      \forall m.\forall o.\forall u.\forall l.\ 
      (\semstep{m}{o}{u}{l}) \Longrightarrow (\sembigstep{m}{o}{u}{l})
    \end{math}
  \end{center}
\end{theorem}

\section{From Modular to Inlining Semantics}

On this section, we present the proof of \refthm{thm-modtoinl}. The
proof exactly follows the definition of $\inlineF{\cdot}$. In other
words, we present the correctness lemma from $(\concatsymb)$ to
$\inlineF{\cdot}$ gradually.

\paragraph{Correctness of the concatenation operator}

Since the concatenation operator $(\concatsymb)$ merely appends two
actions, the correctness lemma also states semantic merge.

\begin{lemma}
  \label{lem-concatsymb}
  For every $o, a_1, u_1, cs_1, v_1, a_2, u_2, cs_2,$ and $v_2,$
  \begin{center}
    \begin{math}
      \begin{array}{l}
        \semact{o}{a_1}{u_1}{cs_1}{v_1} \to \\
        \semact{o}{(\lambda x.a_2)\ v_1}{u_2}{cs_2}{v_2} \to \\
        \semact{o}{\concataction{a_1}{a_2}}{\splus{u_1}{u_2}}
               {\lplus{cs_1}{cs_2}}{v_2}
      \end{array}
    \end{math}
  \end{center}
\end{lemma}
\begin{proof}
  Straightforward by the inductive definition of $(\concatsymb)$.
\end{proof}

Note that two updates $u_1$ and $u_2$ and two called methods $cs_1$
and $cs_2$ do not need to be disjoint each other. This is because the
semantics does not prohibit double writes or calls. The conditions are
defined as one of the well-formedness conditions, which is independent
to the semantics.

TODO: Wf condition on proper sites.

\begin{lemma}[Correctness of the method-inlining operation w.r.t. actions -- intact case]
  \label{lem-inlinedm-action-intact}
  \mbox{}\\
  For every $o, a_f, a, u, cs, v_r,$ and $f,$
  \begin{center}
    \begin{math}
      \begin{array}{l}
        \semact{o}{a}{u}{cs}{v_r} \to \\
        f \notin cs \to \\
        \semact{o}{\inlinedm{a}{(f, a_f)}}{u}{cs}{v_r}
      \end{array}
    \end{math}
  \end{center}
\end{lemma}
\begin{proof}
\end{proof}

\begin{lemma}[Correctness of the method-inlining operation w.r.t. actions -- call case]
  \label{lem-inlinedm-action-call}
  \mbox{}\\
  For every $o, a_f, a, u_f, u, cs_f, cs, v_r, f, v_a, v_r,$ and $v_f,$
  \begin{center}
    \begin{math}
      \begin{array}{l}
        \semact{o}{a}{u}{cs}{v_r} \to \\
        \sdisj{u_f}{u} \to \mdisj{cs_f}{cs} \to \\
        cs[f] = (v_a, v_f) \to \\
        \semact{o}{(\lambda x.a_f)\ v_a}{u_f}{cs_f}{v_f} \to \\
        \semact{o}{\inlinedm{a}{(f, a_f)}}{\splus{u_f}{u}}{\mplus{cs_f}{cs}}{v_r}
      \end{array}
    \end{math}
  \end{center}
\end{lemma}
\begin{proof}
\end{proof}

\begin{lemma}[Correctness of the method-inlining operation w.r.t. \Substep{} -- intact case]
  \label{lem-inlinedmm-substep-intact}
  \mbox{}\\
  For every $m, o, u, ds, cs,$ and $f,$
  \begin{center}
    \begin{math}
      \begin{array}{l}
        \semsstep{m}{o}{u}{\alpha}{ds}{cs} \to \\
        f \notin ds \to \\
        \semsstep{\inlinedmm{m}{f}}{o}{u}{\alpha}{ds}{cs}
      \end{array}
    \end{math}
  \end{center}
\end{lemma}
\begin{proof}
\end{proof}

\begin{lemma}[Correctness of the method-inlining operation w.r.t. \Substep{} -- call case]
  \label{lem-inlinedmm-substep-call}
  \mbox{}\\
  For every $m, o, u_f, u, ds, cs_f, cs, f, a_f, v_a,$ and $v_r,$
  \begin{center}
    \begin{math}
      \begin{array}{l}
        \semsstep{m}{o}{u}{\alpha}{ds}{cs} \to \\
        \semact{o}{(\lambda x.a_f)\ v_a}{u_f}{cs_f}{v_r} \to \\
        \sdisj{u_f}{u} \to \mdisj{cs_f}{cs} \to \\
        cs[f] = (v_a, v_r) \to \\
        \semsstep{\inlinedmm{m}{f}}{o}{\splus{u_s}{u}}
                 {\alpha}{ds}{\mapfilt{\mplus{cs_s}{cs}}{f}}
      \end{array}
    \end{math}
  \end{center}
\end{lemma}
\begin{proof}
\end{proof}

\begin{lemma}[Correctness of the method-inlining operation -- intact case]
  \label{lem-inlinedmm-intact}
  \mbox{}\\
  For every $m, o, u, l,$ and $f,$
  \begin{center}
    \begin{math}
      \begin{array}{l}
        \semsss{m}{o}{u}{l} \to \\
        f \notin (\callsof{l}) \to \\
        \semsss{\inlinedmm{m}{f}}{o}{u}{l}
      \end{array}
    \end{math}
  \end{center}
\end{lemma}
\begin{proof}
\end{proof}

\begin{lemma}[Correctness of the method-inlining operation -- rule case]
  \label{lem-inlinedmm-rule}
  \mbox{}\\
  For every $m, o, u_s, u, ds, cs_s, cs, s,$ and $f,$
  \begin{center}
    \begin{math}
      \begin{array}{l}
        \semsstep{m}{o}{u_s}{\alpharule{s}}{\emptymap}{cs_s} \to \\
        \semsss{m}{o}{u}{\semlbl{\alphameth}{ds}{cs}} \to \\
        f \in cs_s \to f \in ds \to \sdisj{u_s}{u} \to \mdisj{cs_s}{cs} \to \\
        \semsss{\inlinedmm{m}{f}}{o}{\splus{u_s}{u}}{\semlbl{\alpharule{s}}
          {\mapfilt{ds}{f}}{\mapfilt{\mplus{cs_s}{cs}}{f}}}
      \end{array}
    \end{math}
  \end{center}
\end{lemma}
\begin{proof}
\end{proof}

\begin{lemma}[Correctness of the method-inlining operation -- method case 1]
  \label{lem-inlinedmm-meth1}
  \mbox{}\\
  For every $m, o, u_f, u, \alpha, ds, cs_f, cs, g, v_a, v_r,$ and $f,$
  \begin{center}
    \begin{math}
      \begin{array}{l}
        \semsstep{m}{o}{u_f}{\alphameth}{\stupd{}{g}{(v_a, v_r)}}{cs_f} \to \\
        \semsss{m}{o}{u}{\semlbl{\alpha}{ds}{cs}} \to \\
        f \in cs_f \to f \in ds \to g \notin ds \to \\
        \sdisj{u_f}{u} \to \mdisj{cs_f}{cs} \to \\
        \semsss{\inlinedmm{m}{f}}{o}{\splus{u_f}{u}}{\semlbl{\alpha}
          {\mapfilt{\mplus{\stupd{}{g}{(v_a, v_r)}}{ds}}{f}}
          {\mapfilt{\mplus{cs_f}{cs}}{f}}}
      \end{array}
    \end{math}
  \end{center}
\end{lemma}
\begin{proof}
\end{proof}

\begin{lemma}[Correctness of the method-inlining operation -- method case 2]
  \label{lem-inlinedmm-meth2}
  \mbox{}\\
  For every $m, o, u_f, u, \alpha, ds, cs_f, cs, v_a, v_r,$ and $f,$
  \begin{center}
    \begin{math}
      \begin{array}{l}
        \semsstep{m}{o}{u_f}{\alphameth}{\stupd{}{f}{(v_a, v_r)}}{cs_f} \to \\
        \semsss{m}{o}{u}{\semlbl{\alpha}{ds}{cs}} \to \\
        f \in cs_f \to f \notin ds \to \\
        \sdisj{u_f}{u} \to \mdisj{cs_f}{cs} \to \\
        \semsss{\inlinedmm{m}{f}}{o}{\splus{u_f}{u}}{\semlbl{\alpha}{ds}
          {\mapfilt{\mplus{cs_f}{cs}}{f}}}
      \end{array}
    \end{math}
  \end{center}
\end{lemma}
\begin{proof}
\end{proof}

\begin{lemma}[Correctness of the method-inlining operation]
  \label{lem-inlinedmm}
  \mbox{}\\
  For every $m, o, u, l,$ and $f,$
  \begin{center}
    \begin{math}
      \begin{array}{l}
        \semsss{m}{o}{u}{l} \to \\
        f \notin (\callsof{l}) \vee (\defsof{l})[f] = (\callsof{l})[f] \to \\
        \semsss{\inlinedmm{m}{f}}{o}{u}{\hidemeth{l}{f}}
      \end{array}
    \end{math}
  \end{center}
\end{lemma}
\begin{proof}
\end{proof}

\begin{lemma}[Correctness of the methods-inlining operation w.r.t. \hidemethssym{}]
  \label{lem-inlinedmssubo}
  \mbox{}\\
  For every $m, o, u, l,$ and $fs,$
  \begin{center}
    \begin{math}
      \begin{array}{l}
        \semsss{m}{o}{u}{l} \to \\
        \wellhidden{m}{(\hide{l})} \to \\
        \semsss{\inlinedms{m}{fs}}{o}{u}{\hidemeths{l}{fs}}
      \end{array}
    \end{math}
  \end{center}
\end{lemma}
\begin{proof}
\end{proof}

\begin{lemma}[Relation between \hidemethssym{} and \hidesym{}]
  \label{lem-hidemeths}
  \mbox{}\\
  For every $l$ and $fs$,
  \begin{center}
    \begin{math}
      \begin{array}{l}
        \defsof{l} \subseteq fs \to \hidemeths{l}{fs} = \hide{l}
      \end{array}
    \end{math}
  \end{center}
\end{lemma}
\begin{proof}
\end{proof}

\begin{lemma}[Correctness of the methods-inlning operation w.r.t. \Substeps{}]
  \label{lem-inlinedmssub}
  \mbox{}\\
  For every $m, o, u, l,$ and $fs,$
  \begin{center}
    \begin{math}
      \begin{array}{l}
        \semsss{m}{o}{u}{l} \to \\
        \wellhidden{m}{(\hide{l})} \to \\
        \semsss{\inlinedms{m}{fs}}{o}{u}{\hide{l}}
      \end{array}
    \end{math}
  \end{center}
\end{lemma}
\begin{proof}
\end{proof}

\begin{lemma}[Correctness of the methods-inlining operation]
  \label{lem-inlinedms}
  \mbox{}\\
  For every $m, o, u, l,$ and $fs,$
  \begin{center}
    \begin{math}
      \begin{array}{l}
        (\semstep{m}{o}{u}{l}) \to (\semstep{(\inlinedms{m}{fs})}{o}{u}{l})
      \end{array}
    \end{math}
  \end{center}
\end{lemma}
\begin{proof}
\end{proof}

\begin{lemma}[Correctness of $\inline{\cdot}$]
  \label{lem-inline}
  For every $m, o, u,$ and $l,$
  \begin{center}
    \begin{math}
      \begin{array}{l}
        (\semstep{m}{o}{u}{l}) \to (\semstep{\inline{m}}{o}{u}{l})
      \end{array}
    \end{math}
  \end{center}
\end{lemma}
\begin{proof}
\end{proof}

\begin{lemma}[Correctness of $\inlineF{\cdot}$]
  \label{lem-inlinef}
  For every $m, o, u,$ and $l,$
  \begin{center}
    \begin{math}
      \begin{array}{l}
        (\semstep{m}{o}{u}{l}) \to (\semstep{\inlineF{m}}{o}{u}{l})
      \end{array}
    \end{math}
  \end{center}
\end{lemma}
\begin{proof}
\end{proof}

\section{From Modular to Big-step Semantics}



