\chapter{Introduction}

Hardware components have been known to be extremely complex due to
concurrency. A system having concurrency implies that independent
works can be done simultaneously, and it usually occurs as a name of
optimization. For example, instruction pipelining is one of a
representative optimizations, which employes an instruction-level
parallelism to handle multiple executions at the same time.

Modularity has been considered as an effective way to design and
understand such complex hardware components. Modularity indicates that
a complex hardware component can be constructed by composing simple
modules. Several Hardware Description Languages (HDLs) such as Verilog
employes the notion of modules, each of which acts as a simple unit
component.

Among various HDLs, \Bluespec{}~\cite{bsdef, bsref} also allows to
design hardware based not only on modularity, but also the prevalent
paradigm called Guarded Atomic Actions (GAAs). Even though a certain
language allows to design hardware in a modular manner, hardware
design itself inherently has a timing problem... \todo{rephrase}.

Following the concepts of modularity and GAA, we have been defining a
framework called \Kami{}, which is for specifying, verifying,
synthesizing \Bluespec{}-style hardware components. \Kami{} presents a
domain specific language similar to Bluespec, and aims to prove the
correctness of hardware automatically. In order for correctness proof,
we also have defined formal semantics of the \Kami{} language, which
is based on modularity concept. The framework has been built on the
Coq proof assistant.

Modularity fits for scalable verification, especially with theorem
provers. In terms of verification, modularity implies that complex
hardware components can be verified if each simple element is
verified. And in many cases, we reuse such simple components for
different uses. From the perspective of theorem proving, reusing a
hardware component indicates that we can also reuse its correctness
proof. Thanks to the modular semantics in \Kami{}, it is indeed
possible to use the proof of a component whenever it is used for the
part of a design.

An important aspect of the modular semantics is that modules
communicate by labels. It is originated from the concept called
Labeled Transition System (LTS). Each module changes its internal
state and the corresponding label is generated during the state
transition. On the modular semantics, labels are method calls. In
other words, modules communicate by calling methods of target modules.
The modular semantics also take advantage of the LTS concept to define
behaviors of \emph{open systems}. Open systems have external
interactions, which cannot be figured out unless it is composed with
some other modules which can respond to them.

However, the modular semantics has an inherent drawback that it is hard
to infer internal state changes by internal communications.

\todo{an intuitive example, some following paragraphs.}

A number of semantics, which do not have such a drawback, have been
defined but none of them was defined for open systems. In other words,
there have been no semantics which is able to define communication
with external modules. \todo{related works; references}

Hence, in this thesis, I present two additional semantics for open
systems, none of which have the drawback. The first semantics use
static inlining to substitute internal calls to their method
bodies. This operator erases all internal calls in a module, so that
we do not need to care about internal communications. The second
semantics are like typical big-step semantics for C-like
languages. Semantics of internal calls are defined to substitute the
method bodies, and continue judements for the bodies. The big-step
semantics can be regarded as \emph{a dynamic version} of the inlining
semantics.

Implications among semantics are also formally proven, thus
verification developers can use all semantics by converting
them. Semantics does not have equal capabilities. We give two main
implication proofs: 1) the modular semantics imply the inlining
semantics, and 2) the modular semantics imply the big-step semantics.

To sum up, the main contributions of this thesis are:
\begin{itemize}
\item To define two consistent formal semantics for open hardware
  systems.
\item To prove implications among the modular, inlining, and
  operational semantics.
\end{itemize}

\paragraph{Overview}

The thesis is organized as follows: \refchap{chap-backgrounds}
explains a number of prerequisites to understand the hardware design
concept of \Bluespec{}. Related works are also provided to compare
previous approaches to define hardware
semantics. \refchap{chap-semantics} presents the three semantics used
in the \Kami{} framework: modular, inlining, and big-step
semantics. \refchap{chap-implications} compares capability of each
semantics and presents implication proofs among semantics. Lastly, we
draw conclusions and give future works in \refchap{chap-conclusions}.





