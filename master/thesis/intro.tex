\chapter{Introduction}

Modularity has been considered as an effective way to design complex
hardware.

(TODO: necessary?) Bluespec~\cite{bsdef, bsref} allows to design
hardware based on modularity, by using the prevalent paradigm called
``guarded atomic actions.''

Modularity also fits for scalable verification, especially with
theorem provers.

Following such concepts, we have defined the framework Kami, which is
to specify and verify Bluespec-style hardware components.

However, modular verification has an inherent drawback that it is hard
to infer internal state changes by internal communications.

(TODO: an intuitive example, some following paragraphs)

A number of semantics, which do not have such a drawback, have been
defined but all of them lack modularity.

Hence, in this thesis, I present two semantics for open systems
(inlining and operational semantics), none of which has the drawback.

Consistency among semantics are also formally proven, thus
verification developers can use all semantics by converting them.

The main contributions of this thesis are:
\begin{itemize}
\item To define two consistent formal semantics for open hardware
  systems.
\item To prove consistency among the modular, inlining, and
  operational semantics.
\end{itemize}

\paragraph{Overview} The thesis is organized as follows:





