\chapter{Introduction}

Hardware components have been known to be extremely complex due to
concurrency. A system having concurrency implies that independent
works can be done simultaneously, and it usually occurs as a name of
optimization. For example, instruction pipelining is one of a
representative optimizations, which employes an instruction-level
parallelism to handle multiple executions at the same time.

Complex hardware systems are usually design first with Hardware
Description Languages (HDLs).

Efficiency of verifying such hardware designs is deeply related to
formal language definitions.

Bluespec~\cite{bsdef, bsref} allows to design hardware based on
modularity, by using the prevalent paradigm called ``guarded atomic
actions.'' ... Modularity has been considered as an effective way to
design complex hardware.

Modularity also fits for scalable verification, especially with
theorem provers.

Following such concepts, we have defined a framework Kami, which is
for specifying, verifying, synthesizing Bluespec-stype hardware
components.

However, modular verification has an inherent drawback that it is hard
to infer internal state changes by internal communications.

(TODO: an intuitive example, some following paragraphs)

A number of semantics, which do not have such a drawback, have been
defined but all of them lack modularity. ... In other words, such
semantics could not define communication with external modules.

Hence, in this thesis, I present two additional semantics for open
systems (inlining and operational semantics), none of which has the
drawback.

Implications among semantics are also formally proven, thus
verification developers can use all semantics by converting them.

The main contributions of this thesis are:
\begin{itemize}
\item To define two consistent formal semantics for open hardware
  systems.
\item To prove consistency among the modular, inlining, and
  operational semantics.
\end{itemize}

\paragraph{Overview} The thesis is organized as follows:





