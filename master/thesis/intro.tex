\chapter{Introduction}

Hardware components are extremely complex. The main reasons for the
complexity are concurrency and the sheer size of components. For
example, realistic processors make use of instruction pipelining to
execute multiple instructions concurrently. Complexity comes with such
executions, since one must ensure that the concurrent executions
behave as if they are executed in order. The size of components also
grows by complexity. More realistic processors require additional
hardware components such as a reorder buffer. The logic is further
complicated with these components.

Modularity has been considered as an effective way to design and
understand such complex hardware components. A complex hardware
component can be constructed by composing several simple modules. The
notion of modules is so ingrained that all commercial hardware
description languages (HDLs) like Verilog, VHDL, Bluespec, and
Chisel~\cite{verilog, vhdl, bsdef, chisel} support the notion of
modules.

Among various HDLs, \Bluespec{}~\cite{bsdef, bsref} allows designers
to develop hardware not only based on modularity, but also based on
the notion of Guarded Atomic Actions (GAAs)~\cite{daniel-gaa}. Even
though most HDLs allow users to design hardware in a modular manner,
modules cannot be independently designed if they are entangled with
clock-timing issues. For instance, when two modules, a producer of a
value and a consumer of that value, are connected with wires, and the
producer is optimized to use fewer clock cycles to produce the value,
then the consumer should also be modified to consume the values
faster; failure to do so will result in an incorrect overall design.

Following the concepts of modularity and GAA, we have been defining a
framework called \Kami{}~\cite{kami-web, murali-thesis}, which is for
specifying, verifying, and synthesizing \Bluespec{}-style hardware
components. \Kami{} presents a domain specific language similar to
Bluespec. In addition, we define the formal semantics of the \Kami{}
language to enable proving properties of hardware. The framework has
been built on the Coq proof assistant; hence, verification can be
performed by mechanized proofs with a high degree of automation.

The notion of modularity used in designing hardware can be extended to
verification. For modular verification, we want to be able to verify
the individual modules of a large system and compose their proofs to
verify the full system. In many cases, we reuse such simple components
in different larger designs. From the perspective of proof, reusing a
hardware component implies that we can also reuse its related
proofs. Thanks to the modular semantics in \Kami{}, it is indeed
possible to use the proof of a component whenever it is used in a
larger design.

Modules communicate with other modules by calling each other's
methods. In our modular semantics, we model these communications using
\emph{labels}. The semantics of these communicating modules is similar
to those of labeled transition systems (LTS)~\cite{lts}. Each module
changes its internal state and potentially calls methods of other
modules during its state transition. These method calls become labels
in our semantics. Modular semantics, therefore, also defines behaviors
of \emph{open systems}. Open systems have external interactions, whose
specifics cannot be determined unless they are connected with other
modules which can respond to them.

However, modular semantics has an inherent weakness in that it is hard
to infer internal changes. Even if modular semantics allow us to
consider individual modules, we eventually should be able to
understand a part of a design which is still composed of several
modules, when the part cannot be further decomposed. Modular semantics
defines behaviors of such modules by giving the behaviors of each
small module. In this case, we should consider \emph{all possible
  combinations} of the small module behaviors. However, we certainly
do not want to consider all such cases, since we know only a few cases
are feasible by the design itself.

Hence, in this thesis, I present a new semantic approach that is based
on inlining. Inlining semantics is defined for open hardware systems
and resolves the weakness by construction. The semantics uses a static
inlining operation to substitute internal calls to their method
bodies. This operator erases all internal calls in a module, so that
we do not need to care about internal communications.

An implication from modular semantics to inlining semantics is also
formally proven. Thus, it can be used in order to efficiently prove
properties of hardware. Since the two semantics do not have equal
capabilities, we give the implication proof saying that the modular
semantics imply the inlining semantics.

To sum up, the main contributions of this thesis are:
\begin{itemize}
\item To define a new semantic approach based on inlining, for open
  hardware systems.
\item To prove the implication from the modular semantics to the
  inlining semantics.
\end{itemize}

\paragraph{Overview}

The thesis is organized as follows: \refchap{chap:backgrounds}
introduces a number of prerequisites to understand the hardware design
concept of \Bluespec{}. Related works are also provided to compare
previous approaches to define hardware semantics.
\refchap{chap:semantics} presents the two semantics used in the
\Kami{} framework: modular and inlining
semantics. \refchap{chap:implication} compares capability of each
semantics and presents the implication proof between them. Lastly, we
draw conclusions and give future works in \refchap{chap:conclusions}.

