% $Log: abstract.tex,v $
% Revision 1.1  93/05/14  14:56:25  starflt
% Initial revision
% 
% Revision 1.1  90/05/04  10:41:01  lwvanels
% Initial revision
% 
%
%% The text of your abstract and nothing else (other than comments) goes here.
%% It will be single-spaced and the rest of the text that is supposed to go on
%% the abstract page will be generated by the abstractpage environment.  This
%% file should be \input (not \include 'd) from cover.tex.

Hardware components have been known to be extremely complex due to
concurrency. Modularity has been considered as an effective way to
design and understand such complex hardware components. Among various
Hardware Description Languages, \Bluespec{} allows designers to
develop hardware based not only on modularity, but also on the
prevalent paradigm called Guarded Atomic Actions (GAAs). Following the
concept of \Bluespec{}, we have been defining a framework
called \Kami{}, which is for verifying \Bluespec{}-style hardware
designs. Modularity also fits for verifying hardware components,
thanks to the Labeled Transition System (LTS) concept. However,
modular verification has an inherent weakness that it is difficult to
infer internal state changes by internal communications. In this
thesis, I present a new semantic approach, which is based on
inlining. Inlining semantics is defined for open hardware systems, and
resolve the weakness by construction. An implication from the modular
semantics to the inlining semantics is also formally proven, thus it
can be used in order to efficiently prove hardware properties.

