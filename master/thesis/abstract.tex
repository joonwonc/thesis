% $Log: abstract.tex,v $
% Revision 1.1  93/05/14  14:56:25  starflt
% Initial revision
% 
% Revision 1.1  90/05/04  10:41:01  lwvanels
% Initial revision
% 
%
%% The text of your abstract and nothing else (other than comments) goes here.
%% It will be single-spaced and the rest of the text that is supposed to go on
%% the abstract page will be generated by the abstractpage environment.  This
%% file should be \input (not \include 'd) from cover.tex.

Hardware components are extremely complex due to concurrency.
Modularity has been considered as an effective way to design and
understand such complex hardware components. Among various hardware
description languages (HDLs), \Bluespec{} allows designers to develop
hardware not only based on modularity, but also based on the notion of
guarded atomic actions (GAAs). Following the concepts of modularity
and GAA, we have been defining a framework called \Kami{} to specify,
verify, and synthesize \Bluespec{}-style hardware components. However,
modular semantics has an inherent weakness in that it is hard to infer
internal changes. In this thesis, I present a new semantic approach
based on inlining. Inlining semantics is defined for open hardware
systems and resolves the weakness by construction.  An implication
from modular semantics to inlining semantics is also formally proven;
thus the inlining semantics can be used to efficiently prove hardware
properties.

