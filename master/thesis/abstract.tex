% $Log: abstract.tex,v $
% Revision 1.1  93/05/14  14:56:25  starflt
% Initial revision
% 
% Revision 1.1  90/05/04  10:41:01  lwvanels
% Initial revision
% 
%
%% The text of your abstract and nothing else (other than comments) goes here.
%% It will be single-spaced and the rest of the text that is supposed to go on
%% the abstract page will be generated by the abstractpage environment.  This
%% file should be \input (not \include 'd) from cover.tex.

Hardware components have been known to be extremely complex due to
concurrency. Modularity has been considered as an effective way to
design and understand such complex hardware components. Among various
Hardware Description Languages (HDLs), \Bluespec{} allows to design
hardware based not only on modularity, but also on the prevalent
paradigm called Guarded Atomic Actions (GAAs). Following the concepts
of \Bluespec{}, we have been defining a framework called \Kami{},
which is to verify \Bluespec{}-style hardware components. Modularity
also fits for verifying hardware components, thanks to the Labeled
Transition System (LTS) concept. However, modular verification has an
inherent weakness that it is hard to infer internal state changes by
internal communications. Hence, in this thesis, I present an
additional semantics definition, which is based on inlining. The
inlining semantics is defined for open hardware systems and resolve
the weakness by construction. An implication from the modular
semantics to the inlining semantics is also formally proven, thus it
can be used to handle the semantics efficiently.

