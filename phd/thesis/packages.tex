%% Packages --

\usepackage{xcolor}
\definecolor{mordantred19}{rgb}{0.68, 0.05, 0.0}
\definecolor{forestgreen}{rgb}{0.13, 0.55, 0.13}
\definecolor{mediumtealblue}{rgb}{0.0, 0.33, 0.71}
\definecolor{dimgray}{rgb}{0.41, 0.41, 0.41}
\definecolor{brickred}{rgb}{0.8, 0.25, 0.33}
\definecolor{htmlcssgreen}{rgb}{0.0, 0.5, 0.0}

\colorlet{myred}{mordantred19}
\colorlet{mygreen}{forestgreen}
\colorlet{myblue}{mediumtealblue}
\colorlet{mygray}{dimgray}

\usepackage{geometry}
\usepackage{booktabs}
\usepackage[labelfont=bf]{caption}
\usepackage{subcaption}
\usepackage{mathtools}
\usepackage{amsmath}
\usepackage{amssymb}
\usepackage{amsthm}
\usepackage{semantic}
\usepackage{stmaryrd}
\usepackage{hyperref}
\hypersetup{
  colorlinks=true,
  linkcolor=brickred,
  citecolor=htmlcssgreen
}
\usepackage{tikz}
\usetikzlibrary{arrows,decorations.markings}
\pgfdeclarelayer{bg} % declare background layer
\pgfsetlayers{bg,main} % set the order of the layers (main is the standard layer)

\usepackage{enumitem}
\usepackage{listings}
\usepackage{url}
\def\UrlBreaks{\do\/\do-}
%% \usepackage{wrapfig}
\usepackage{multirow}
\usepackage{hhline}
\usepackage{adjustbox}
\usepackage{ifthen}
\usepackage{colortbl}

\lstset{
  basicstyle=\ttfamily\footnotesize,
  columns=fullflexible,
  emph={fun, assert, Definition, Rule, rule, Endrule, endrule, using, template, accepts, holding, release, from, me, requires, transition, ltac},
  emphstyle={\bfseries\color{mygreen}},
  %% backgroundcolor=\color{backcolour},
  %% commentstyle=\color{codegreen},
  %% keywordstyle=\color{magenta},
  %% numberstyle=\tiny\color{codegray},
  %% stringstyle=\color{codepurple},
  breakatwhitespace=false,
  %% breaklines=true,
  %% captionpos=b,
  belowcaptionskip=8pt,
  keepspaces=true,
  numbers=left,
  numbersep=5pt,
  showspaces=false,
  showstringspaces=false,
  showtabs=false,
  tabsize=2,
  frame=single,
  xleftmargin=0.03\columnwidth,
  xrightmargin=0.03\columnwidth,
}
\def\slstinline{\lstinline[basicstyle=\ttfamily\small]}

\theoremstyle{plain}
\newtheorem{theorem}{Theorem}[section]
\newtheorem{conjecture}[theorem]{Conjecture}
\newtheorem{proposition}[theorem]{Proposition}
\newtheorem{lemma}[theorem]{Lemma}
\newtheorem{corollary}[theorem]{Corollary}
\newtheorem{example}[theorem]{Example}
\newtheorem{definition}[theorem]{Definition}

%% -- End of Packages
