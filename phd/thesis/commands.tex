%% Commands --

%% General commands
\newcommand{\todo}[1]{\emph{\textcolor{red}{TODO: #1}}}
\newcommand{\note}[1]{\emph{\textcolor{blue}{NOTE: #1}}}
\newcommand{\panic}[1]{\textbf{\textcolor{red}{(#1)}}}

\renewcommand{\subsectionautorefname}{section}
\renewcommand{\subsubsectionautorefname}{section}

\newcommand{\ie}{i.e.,}
\newcommand{\eg}{e.g.,}
\newcommand{\aka}{a.k.a.}

\newcommand{\tuple}[1]{\langle #1 \rangle}
\renewcommand\qedsymbol{$\blacksquare$}

\newcommand{\cmark}{\ding{51}}
\newcommand{\xmark}{\ding{55}}

\newcommand{\temph}[1]{\underline{#1}}

%% Hemiola

\newcommand{\hemiola}{Hemiola}

\renewcommand{\listof}[1]{\ensuremath{\overline{#1}}}
\newcommand{\llistof}[1]{\ensuremath{\overline{\overline{#1}}}}
\newcommand{\listtof}[1]{\ensuremath{\overline{#1}}}
\newcommand{\listnil}{\ensuremath{\lbrack\rbrack}}
\newcommand{\listcons}[2]{\ensuremath{#1 + #2}}
\newcommand{\listsingle}[1]{\ensuremath{\lbrack #1 \rbrack}}
\newcommand{\listapp}[2]{\ensuremath{#1 + #2}}
\newcommand{\listsub}[2]{\ensuremath{#1 - #2}}
\newcommand{\listdisj}[2]{\ensuremath{#1 \; \# \; #2}}
\newcommand{\listconcat}[1]{\ensuremath{\oplus #1}}
\newcommand{\sizeof}[1]{\ensuremath{| #1 |}}

\newcommand{\mapupd}[3]{\listapp{#1}{(#2, #3)}}
\newcommand{\mapupds}[2]{\listapp{#1}{#2}}
\newcommand{\mapsubs}[2]{\listsub{#1}{#2}}

%% Stretchable squiggly arrows
%% cf. https://tex.stackexchange.com/questions/75669/how-do-i-put-text-over-a-squiggly-arrow
\newcommand{\squig}{{\scriptstyle\sim\mkern-3.9mu}}
\newcommand{\lsquigend}{{\scriptstyle\lhd\mkern-3mu}}
\newcommand{\rsquigend}{{\scriptstyle\rule{.1ex}{0ex}\rhd}}
\newcounter{sqindex}
\newcommand\squigs[1]{%
  \setcounter{sqindex}{0}%
  \whiledo {\value{sqindex}< #1}{\addtocounter{sqindex}{1}\squig}%
}
\newcommand\rsquigarrow[2]{%
  \mathbin{\stackon[2pt]{\squigs{#2}\rsquigend}{\scriptscriptstyle\text{#1\,}}}%
}
\newcommand\lsquigarrow[2]{%
  \mathbin{\stackon[2pt]{\lsquigend\squigs{#2}}{\scriptscriptstyle\text{\,#1}}}%
}

\newcommand{\idxOf}[1]{\ensuremath{#1.i}}

\newcommand{\boolt}{\ensuremath{\mathbb{B}}}
\newcommand{\propt}{\ensuremath{\mathbb{P}}}
\newcommand{\hidt}{\ensuremath{\mathbb{I}}}
\newcommand{\hidxt}{\ensuremath{\mathbb{I}}}
\newcommand{\hvaluet}{\ensuremath{\mathbb{V}}}
\newcommand{\hmsgt}{\ensuremath{\mathbb{M}}}
\newcommand{\hostt}{\ensuremath{\mathbb{O}}}
\newcommand{\hidmt}{\ensuremath{\mathbb{IM}}}
%% \newcommand{\hprect}{\ensuremath{\mathcal{P}}}
%% \newcommand{\htrst}{\ensuremath{\mathcal{T}}}

\newcommand{\hsysIn}[1]{\ensuremath{\listof{#1_{\textrm{in}}}}}
\newcommand{\hsysRq}[1]{\ensuremath{\listof{#1_{\textrm{rq}}}}}
\newcommand{\hsysRs}[1]{\ensuremath{\listof{#1_{\textrm{rs}}}}}
\newcommand{\hsys}[4]{\ensuremath{\tuple{#1, #2, #3, #4}}}
\newcommand{\hsyss}[2]{\hsys{#1}{\hsysIn{#2}}{\hsysRq{#2}}{\hsysRs{#2}}}
\newcommand{\hsysInA}[1]{\ensuremath{#1.\hsysIn{i}}}
\newcommand{\hsysRqA}[1]{\ensuremath{#1.\hsysRq{i}}}
\newcommand{\hsysRsA}[1]{\ensuremath{#1.\hsysRs{i}}}

\newcommand{\hsysobjs}[1]{\ensuremath{#1.\listof{O}}}
\newcommand{\hobjrules}[1]{\ensuremath{#1.\listof{r}}}

\newcommand{\msgbuild}[3]{\tuple{#1, #2, #3}}
\newcommand{\msgty}{\ensuremath{\textsf{ty}}}
\newcommand{\msgid}{\ensuremath{\textsf{id}}}
\newcommand{\msgval}{\ensuremath{\textsf{val}}}
\newcommand{\mtypeOf}[1]{#1.\msgty{}}
\newcommand{\midOf}[1]{#1.\msgid{}}
\newcommand{\mvalOf}[1]{#1.\msgval{}}
\newcommand{\idmbuild}[2]{\ensuremath{(#1, #2)}}

\newcommand{\ruleprec}{\ensuremath{\textsf{prec}}}
\newcommand{\ruletrs}{\ensuremath{\textsf{trs}}}
\newcommand{\rprecOf}[1]{#1.\ruleprec{}}
\newcommand{\rtrsOf}[1]{#1.\ruletrs{}}

\newcommand{\msgIns}[1]{\listof{#1^{\textrm{ins}}}}
\newcommand{\msgOuts}[1]{\listof{#1^{\textrm{outs}}}}
\newcommand{\midxIns}[1]{\listof{\idxOf{#1^{\textrm{ins}}}}}
\newcommand{\midxOuts}[1]{\listof{\idxOf{#1^{\textrm{outs}}}}}

\newcommand{\lblEmpty}{\ensuremath{l_\epsilon}}
\newcommand{\lblIns}[1]{\ensuremath{l_{\textrm{in}}(#1)}}
\newcommand{\lblOuts}[1]{\ensuremath{l_{\textrm{out}}(#1)}}
\newcommand{\lblInt}[4]{\ensuremath{l_{\textrm{int}}(#1, #2, #3, #4)}}

\newcommand{\hmpt}{\ensuremath{\mathcal{M}}}
\newcommand{\hsttr}{\ensuremath{\listtof{\hostt} \ast \hmpt}}
\newcommand{\hstt}{\ensuremath{\mathbb{S}}}
\newcommand{\hst}[2]{\ensuremath{\tuple{#1, #2}}}
\newcommand{\hstm}[2]{\ensuremath{\left\langle\begin{array}{l}#1,\\ #2 \end{array}\right\rangle}}

\newcommand{\objInit}[1]{\ensuremath{#1_{\textsf{init}}}}
\newcommand{\sysInit}[1]{\ensuremath{#1_{\textsf{init}}}}

\newcommand{\disj}[2]{\ensuremath{#1 \; \# \; #2}}
\newcommand{\nodup}[1]{\ensuremath{\textsf{List.NoDup}\: #1}}
\newcommand{\heads}[1]{\ensuremath{#1.\textsf{hds}}}

\newcommand{\semstep}[4]{\ensuremath{#2 \xrightarrow[#1]{#3} #4}}
\newcommand{\semsteps}[4]{\ensuremath{#2 \xRightarrow[#1]{#3} #4}}
\newcommand{\semrch}[2]{\ensuremath{#1 \Rightarrow #2}}
\newcommand{\semleg}[2]{\ensuremath{#1 \xRightarrow{#2} \bullet}}
\newcommand{\sembeh}[2]{\ensuremath{#1 \Downarrow #2}}
\newcommand{\behOf}[1]{\ensuremath{\lfloor #1 \rfloor}}

\newcommand{\refines}[2]{\ensuremath{#1 \sqsubseteq #2}}

\newcommand{\enqMsgs}[2]{\mapupds{#1}{#2}}
\newcommand{\deqMsgs}[2]{\mapsubs{#1}{#2}}

%% Serializability

\newcommand{\atomic}[3]{\ensuremath{#1 \stackrel{#2}{\squigs{3}\rsquigend} #3}}
\newcommand{\atomicShort}[1]{\ensuremath{(\stackrel{#1}{\squigs{3}\rsquigend})}}
\newcommand{\atomicLong}[4]{\ensuremath{#1 \stackrel{#2}{\squigs{#3}\rsquigend} #4}}
\newcommand{\extatomic}[4]{\ensuremath{#1 \vdash #2 \stackrel{#3}{\squigs{3}\rsquigend}_{\textsf{ext}} #4}}
\newcommand{\extatomicShort}[2]{\ensuremath{#1 \stackrel{#2}{\squigs{3}\rsquigend}_{\textsf{ext}}}}
\newcommand{\extatomicLong}[5]{\ensuremath{#1 \vdash #2 \stackrel{#3}{\squigs{#4}\rsquigend}_{\textsf{ext}} #5}}
\newcommand{\trsn}[2]{\ensuremath{#1 \sswarrow #2}}
\newcommand{\atomicHsts}[1]{\ensuremath{\overline{\mathcal{A}}\,(#1)}}

\newcommand{\extatomicshort}[2]{\ensuremath{#1 \stackrel{#2}{\rightsquigarrow}_{\textsf{ext}}}}
\newcommand{\intatomicshort}[2]{\ensuremath{#1 \stackrel{#2}{\rightsquigarrow}_{\textsf{int}}}}

\newcommand{\amsgi}[1]{\ensuremath{#1^{\textrm{init}}}}
\newcommand{\amsge}[1]{\ensuremath{#1^{\textrm{end}}}}

\newcommand{\hseq}[2]{\ensuremath{\mathsf{Sequential}\ #1\ #2}}
\newcommand{\hsrzl}[2]{\ensuremath{\mathsf{Serializable}\ #1\ #2}}
\newcommand{\hsrz}[1]{\ensuremath{\mathsf{Serializable}\ #1}}

%% Serializability proof

\newcommand{\strsn}[2]{\ensuremath{#1 \sswarrow_{s} #2}}
\newcommand{\hsseq}[3]{\ensuremath{\mathsf{Sequential}_{\textrm{s}}\ #1\ #2\ #3}}

\newcommand{\hcontsymb}{\ensuremath{\succ}}
\newcommand{\hcont}[2]{\ensuremath{#1 \hcontsymb #2}}
\newcommand{\hextcont}[3]{\ensuremath{#1 \vdash #2 \hcontsymb_{\textsf{ext}} #3}}
\newcommand{\hdiscontsymb}{\ensuremath{\nsucc}}
\newcommand{\hdiscont}[2]{\ensuremath{#1 \hdiscontsymb{} #2}}
\newcommand{\hitlv}[2]{\ensuremath{\mathsf{Interleaved}\ #1\ #2}}

\newcommand{\hmgb}[3]{\ensuremath{#1 \vdash #2 \bowtie #3}}
\newcommand{\smgb}[1]{\ensuremath{\mathsf{Mergeable}\ #1}}

\newcommand{\ontree}[2]{\ensuremath{\mathsf{OnTree}\ #1\ #2}}
\newcommand{\goodrules}[2]{\ensuremath{\mathsf{GoodRules}\ #1\ #2}}

\newcommand{\subtree}[1]{\ensuremath{\textrm{tr}\,(#1)}}
\newcommand{\subtreec}[1]{\ensuremath{\textrm{tr}^{-1}\,(#1)}}
\newcommand{\objsinv}[1]{\ensuremath{\textsf{Invalid}\;(#1)}}

\newcommand{\implcoh}[3]{\ensuremath{\textit{Coh}\,(#1, #2, #3)}}
\newcommand{\speccoh}[1]{\ensuremath{\textit{Spec}\,(v)}}

%% Visualization of messages/histories

\newcommand*\rdrecsm[1]{
  \adjustbox{raise=.1ex, trim=4pt 0pt 0pt 0pt}{
    \tikz[baseline=(char.base)]{
      \node[shape=rectangle, draw, inner sep=0pt, minimum size=7pt, fill=white, text=myred]
      (char) {\scriptsize\sffamily\bf #1};}}
}
\newcommand*\blrecsm[1]{
  \adjustbox{raise=.1ex, trim=4pt 0pt 0pt 0pt}{
    \tikz[baseline=(char.base)]{
      \node[shape=rectangle, draw, inner sep=0pt, minimum size=7pt, fill=white, text=myblue]
      (char) {\scriptsize\sffamily\bf #1};}}
}

\newcommand*\rdcircf[1]{
  \adjustbox{raise=.2ex, trim=4pt 0pt 0pt 0pt}{
    \tikz[baseline=(char.base)]{
      \node[shape=circle, inner sep=0pt, minimum size=8pt, fill=myred, text=white]
      (char) {\tiny\sffamily #1};}}
}
\newcommand*\blcircf[1]{
  \adjustbox{raise=.2ex, trim=4pt 0pt 0pt 0pt}{
    \tikz[baseline=(char.base)]{
      \node[shape=circle, inner sep=0pt, minimum size=8pt, fill=myblue, text=white]
      (char) {\tiny\sffamily #1};}}
}

\newcommand{\rdrecmsg}[2]{{\color{myred}\rdrecsm{#1}{\sf #2}}}
\newcommand{\rdrecmsgsm}[2]{\rdrecmsg{#1}{\small #2}}
\newcommand{\blrecmsg}[2]{{\color{myblue}\blrecsm{#1}{\sf #2}}}
\newcommand{\blrecmsgsm}[2]{\blrecmsg{#1}{\small #2}}

\newcommand{\msgsf}[1]{\ensuremath{\textsf{#1}}}
\newcommand{\msgsfsm}[1]{{\small\ensuremath{\textsf{#1}}}}
\newcommand{\rdmsg}[1]{{\color{myred}\msgsf{#1}}}
\newcommand{\rdmsgsm}[1]{{\color{myred}\msgsfsm{#1}}}
\newcommand{\blmsg}[1]{{\color{myblue}\msgsf{#1}}}
\newcommand{\blmsgsm}[1]{{\color{myblue}\msgsfsm{#1}}}
\newcommand{\idmsf}[2]{\idmbuild{#1}{\textsf{#2}}}
\newcommand{\spredmsg}[3]{\ensuremath{#1 \vdash #2\{#3\}}}
\newcommand{\predmsg}[2]{\ensuremath{#1\{#2\}}}

\newcommand{\rulelbl}[2]{\ensuremath{#1:(#2)}}
\newcommand{\rdrulelbl}[2]{{\small\color{myred}\rulelbl{#1}{#2}}}
\newcommand{\blrulelbl}[2]{{\small\color{myblue}\rulelbl{#1}{#2}}}

\newcommand{\rdcircfl}[2]{\rdcircf{#1}{#2}}
\newcommand{\rdcircflsm}[2]{\rdcircf{#1}{\footnotesize #2}}
\newcommand{\blcircfl}[2]{\blcircf{#1}{#2}}
\newcommand{\blcircflsm}[2]{\blcircf{#1}{\footnotesize #2}}

%% Case studies

\newcommand{\mesi}{\ensuremath{\textsf{MESI}}}
\newcommand{\msi}{\ensuremath{\textsf{MSI}}}
\newcommand{\stM}{\ensuremath{\textsf{M}}}
\newcommand{\stE}{\ensuremath{\textsf{E}}}
\newcommand{\stS}{\ensuremath{\textsf{S}}}
\newcommand{\stI}{\ensuremath{\textsf{I}}}
\newcommand{\stNP}{\ensuremath{\textsf{NP}}}
\newcommand{\dir}[2]{\ensuremath{#1_{\tuple{#2}}}}

%% -- End of Commands

%% TikZ templates --

\tikzset{
  pics/skeleton-pcc/.style n args={3}{
    code = {
      \node at (0, 0.3) {#1};
      \node at (-1.7, -1.7) {#2};
      \node at (1.7, -1.7) {#3};
      % between P and C_1
      \draw [line width=0.2mm, <-<] (-0.6, -0.3) -- (-1.6, -1.3);
      \draw [line width=0.2mm, <-<] (-0.4, -0.3) -- (-1.4, -1.3);
      \draw [line width=0.2mm, >->] (-0.2, -0.3) -- (-1.2, -1.3);
      % between P and C_2
      \draw [line width=0.2mm, >->] (0.6, -0.3) -- (1.6, -1.3);
      \draw [line width=0.2mm, <-<] (0.4, -0.3) -- (1.4, -1.3);
      \draw [line width=0.2mm, <-<] (0.2, -0.3) -- (1.2, -1.3);
    }
  },
  skeleton-ex1/.pic = { % C_1 external
    \draw [line width=0.2mm, <-<] (-1.8, -2.0) -- (-1.8, -2.6);
    \draw [line width=0.2mm, >->] (-1.6, -2.0) -- (-1.6, -2.6);
  },
  skeleton-ex2/.pic = { % C_2 external
    \draw [line width=0.2mm, >->] (1.8, -2.0) -- (1.8, -2.6);
    \draw [line width=0.2mm, <-<] (1.6, -2.0) -- (1.6, -2.6);
  },
  skeleton-midx-e1/.pic = {
    % C_1 external
    \node at (-1.8, -2.75) {\tiny $1$};
    \node at (-1.6, -2.75) {\tiny $2$};
  },
  skeleton-midx-e2/.pic = {
    % C_2 external
    \node at (1.8, -2.75) {\tiny $10$};
    \node at (1.6, -2.75) {\tiny $9$};
  },
  skeleton-midx-pc1/.pic = {
    % between P and C_1
    \node at (-0.65, -0.15) {\tiny $3$};
    \node at (-0.45, -0.15) {\tiny $4$};
    \node at (-0.25, -0.15) {\tiny $5$};
  },
  skeleton-midx-pc2/.pic = {
    % between P and C_2
    \node at (0.65, -0.15) {\tiny $8$};
    \node at (0.45, -0.15) {\tiny $7$};
    \node at (0.25, -0.15) {\tiny $6$};
  }
}

\tikzset{
  pics/skeleton-pcce1/.style n args={3}{
    code = {
      \pic at (0, 0) {skeleton-pcc={#1}{#2}{#3}};
      \pic at (0, 0) {skeleton-ex1};
    }
  },
  pics/skeleton-pcce2/.style n args={3}{
    code = {
      \pic at (0, 0) {skeleton-pcc={#1}{#2}{#3}};
      \pic at (0, 0) {skeleton-ex2};
    }
  },
  pics/skeleton-pcce12/.style n args={3}{
    code = {
      \pic at (0, 0) {skeleton-pcc={#1}{#2}{#3}};
      \pic at (0, 0) {skeleton-ex1};
      \pic at (0, 0) {skeleton-ex2};
    }
  },
  pics/skeleton-pc/.style n args={2}{
    code = {
      \node at (0, 0) {#1};
      \node at (0, -1.5) {#2};
      % between P and C
      \draw [<-<] (-0.2, -0.3) -- (-0.2, -1.2);
      \draw [<-<] (0, -0.3) -- (0, -1.2);
      \draw [>->] (0.2, -0.3) -- (0.2, -1.2);
      % C external
      \draw [<-<] (-0.1, -1.8) -- (-0.1, -2.3);
      \draw [>->] (0.1, -1.8) -- (0.1, -2.3);
    }
  }
}

\tikzset{
  spec/.pic = {
    \node at (0, 0) {$\textrm{Spec}(v)$};
    % C_1 external
    \draw [<-<] (-0.4, -0.3) -- (-0.5, -0.8);
    \draw [>->] (-0.2, -0.3) -- (-0.3, -0.8);
    %% \node at (-0.5, -0.95) {\tiny $1$};
    %% \node at (-0.3, -0.95) {\tiny $2$};
    % C_2 external
    \draw [>->] (0.4, -0.3) -- (0.5, -0.8);
    \draw [<-<] (0.2, -0.3) -- (0.3, -0.8);
    %% \node at (0.5, -0.95) {\tiny $10$};
    %% \node at (0.3, -0.95) {\tiny $9$};
  }
}

%% -- End of TikZ templates
