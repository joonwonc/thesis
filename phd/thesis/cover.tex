\title{Structural Design and Proof of\\Hierarchical Cache-Coherence Protocols}

\author{Joonwon Choi}
\prevdegrees{M.S., Massachusetts Institute of Technology (2016)\\B.S., Seoul National University (2013)}
\department{Department of Electrical Engineering and Computer Science}
\degree{Doctor of Philosophy in Electrical Engineering and Computer Science}

\degreemonth{February}
\degreeyear{2021}
\thesisdate{Jan 27, 2021}

%% By default, the thesis will be copyrighted to MIT.  If you need to copyright
%% the thesis to yourself, just specify the `vi' documentclass option.  If for
%% some reason you want to exactly specify the copyright notice text, you can
%% use the \copyrightnoticetext command.
%\copyrightnoticetext{\copyright IBM, 1990.  Do not open till Xmas.}

% If there is more than one supervisor, use the \supervisor command
% once for each.
\supervisor{Adam Chlipala}{Associate Professor of Electrical Engineering and Computer Science}
\supervisor{Arvind}{Professor of Electrical Engineering and Computer Science}

% This is the department committee chairman, not the thesis committee
% chairman.  You should replace this with your Department's Committee
% Chairman.
\chairman{Leslie A. Kolodziejski}{Professor of Electrical Engineering and Computer Science\\Chair, Department Committee on Graduate Students}

% Make the titlepage based on the above information.  If you need
% something special and can't use the standard form, you can specify
% the exact text of the titlepage yourself.  Put it in a titlepage
% environment and leave blank lines where you want vertical space.
% The spaces will be adjusted to fill the entire page.  The dotted
% lines for the signatures are made with the \signature command.
\maketitle

% The abstractpage environment sets up everything on the page except
% the text itself.  The title and other header material are put at the
% top of the page, and the supervisors are listed at the bottom.  A
% new page is begun both before and after.  Of course, an abstract may
% be more than one page itself.  If you need more control over the
% format of the page, you can use the abstract environment, which puts
% the word "Abstract" at the beginning and single spaces its text.

%% You can either \input (*not* \include) your abstract file, or you can put
%% the text of the abstract directly between the \begin{abstractpage} and
%% \end{abstractpage} commands.

% First copy: start a new page, and save the page number.
\cleardoublepage
% Uncomment the next line if you do NOT want a page number on your
% abstract and acknowledgments pages.
% \pagestyle{empty}
\setcounter{savepage}{\thepage}
\begin{abstractpage}
  Cache-coherence protocols have been one of the greatest correctness challenges of the hardware world.
  A memory subsystem usually consists of several caches and the main memory, and a cache-coherence protocol defined in such a system allows multiple memory-access transactions to execute in a distributed manner, across the levels of a cache hierarchy.
  This source of concurrency is the most challenging part in formal verification of cache coherence.

  In this dissertation, we introduce \hemiola{}, a framework embedded in Coq to design, prove, and synthesize cache-coherence protocols in a structural way.
  The framework guides the user to design protocols that never experience inconsistent interleavings while handling transactions concurrently.
  Any protocol designed in \hemiola{} always satisfies the \emph{serializability} property, allowing a user to prove the protocol assuming that \emph{transactions are executed one-at-a-time}.
  The proof relies on conditions on the protocol topology and state-change rules, but we have designed a domain-specific protocol language that guides the user to design protocols that satisfy these properties by construction.

  The framework also provides a novel way to design and prove invariants by adding predicates to messages in the system, called \emph{predicate messages}.
  On top of serializability, it is much simpler to prove a predicate message, since it is guaranteed that the predicate is not spuriously broken by other messages.

  We used \hemiola{} to design and prove hierarchical MSI and MESI protocols, in both inclusive and noninclusive variants, as case studies.
  We also demonstrated that the case-study protocols are indeed hardware-synthesizable, by using a compilation/synthesis toolchain in the framework.
\end{abstractpage}

% Additional copy: start a new page, and reset the page number.  This way,
% the second copy of the abstract is not counted as separate pages.
% Uncomment the next 6 lines if you need two copies of the abstract
% page.
% \setcounter{page}{\thesavepage}
% \begin{abstractpage}
% % $Log: abstract.tex,v $
% Revision 1.1  93/05/14  14:56:25  starflt
% Initial revision
% 
% Revision 1.1  90/05/04  10:41:01  lwvanels
% Initial revision
% 
%
%% The text of your abstract and nothing else (other than comments) goes here.
%% It will be single-spaced and the rest of the text that is supposed to go on
%% the abstract page will be generated by the abstractpage environment.  This
%% file should be \input (not \include 'd) from cover.tex.

Hardware components are extremely complex due to concurrency.
Modularity has been considered as an effective way to design and
understand such complex hardware components. Among various Hardware
Description Languages (HDLs), \Bluespec{} allows designers to develop
hardware not only based on modularity, but also based on the notion of
Guarded Atomic Actions (GAAs). Following the concepts of modularity
and GAA, we have been defining a framework called \Kami{}, which is
for specifying, verifying, and synthesizing \Bluespec{}-style hardware
components. However, modular semantics has an inherent weakness in
that it is hard to infer internal changes. In this thesis, I present a
new semantic approach based on inlining. Inlining semantics is defined
for open hardware systems and resolves the weakness by construction.
An implication from modular semantics to inlining semantics is also
formally proven, thus the inlining semantics can be used to
efficiently prove hardware properties.


% \end{abstractpage}

\cleardoublepage

\section*{Acknowledgments}

When I was in the first year of my PhD, I happened to watch a very popular comic strip (probably only among graduate students) named ``PHD Comics''; this particular comic strip\footnote{PHD Comics: Professed? \url{http://phdcomics.com/comics.php?f=1672}} was composed only of four panels, with captions ``Impressed! Oppressed. Depressed. and Mostly Pressed,'' summarizing a life of a graduate student.
I wondered if my graduate-student life would be like this cartoon.

After 6.5 years of my PhD, now I recall this comic strip in this dissertation and think about it again.
Indeed it was.
I was impressed by brilliant faculties and students who I worked with.
I was oppressed by research challenges that I had to get over.
I was depressed by a number of rejections I got from conferences and the unknown future.
I have felt constant pressure that I should make good research outcomes.

That being said, I have never thought it was a bad decision that I joined MIT as a graduate student.
I am rather grateful that I could do meaningful research in a great environment.
I am also grateful to myself for overcoming all the oppression, depression, and pressure that I have had during my entire PhD study.

I would definitely not have been able to complete my PhD successfully without helps from the people around me.
First of all, I am sincerely grateful to my research advisors Professor Adam Chlipala and Professor Arvind for their invaluable advices.
Before joining MIT, I only had a research background in programming languages and compilers, not much in hardware verification.
I especially thank Arvind for providing me a fast track to learn modern hardware design and verification.
More importantly, I also learned from Arvind that it is really crucial to have a ``core idea'' in one's research career; I have no choice but to mention the one-rule-at-a-time semantics in Bluespec, for instance, that has dominated all my research projects.
My PhD study can be summarized as seeking my own core idea; to a certain degree I am satisfied in that now I have my own philosophy in hardware verification.

I also thank Adam for discussing every small technical detail with me as well as keeping the direction of research projects in the right way.
During my entire PhD, I think I have been mostly impressed by Adam, astonished by his broad knowledge and insight.
Adam has motivated me greatly in the sense that I have thought like at the end of my graduate study I should know more than Adam for what I have studied for my PhD.
(I am still not sure if I have achieved it, though.)
One thing I am sure is that I will not be able to work with anyone who is more quick-witted than Adam, and I regard it as my best luck in my PhD.

I would also like to thank the other thesis committee member, Professor Nickolai Zeldovich.
I have to say that the basic idea of the \hemiola{} framework, serializability, is ``indirectly'' from Nickolai.
When I was in the second year, I took a seminar course by Nickolai and studied a research paper that used the commuting-reduction technique, which motivated me to use a similar technique to verify cache-coherence protocols.
I thank Nickolai for opening such a wonderful course at that time.

Next, I would like to give my gratitude to the MIT PLV group members.
The first project that I was mainly involved with was the Kami hardware-verification framework.
At that time, I had to learn advanced computer architecture as well as working on the framework.
I especially thank Muralidaran Vijayaraghavan to work on Kami with me and to give me intuitions about hardware design and verification in the early phase of development.
I also thank the other student author of Kami, Benjamin Sherman, for building basic Coq library for the framework.
The finite-map library he built has been further developed and used in the \hemiola{} framework as well.

I would also like to thank the members of the Lightbulb project: Andres Erbsen for the broad knowledge of software and hardware, brilliant intuitions about the project, and most practically comprehensive knowledge for the Coq proof assistant.
Also, Samuel Gruetter for great meticulousness in the entire project as well as the broad knowledge of program logic.
I was very happy to work with these students, convincing myself that theorem proving has a decent possibility to be used for verifying realistic computer systems.

My another big gratitude should go to hardware-verification experts in the PLV group: Cl\'ement Pit-Claudel, Stella Lau, and Thomas Bourgeat for listening to the practice talk for my PhD defense and giving me invaluable comments and suggestions.
I want to especially thank Cl\'ement; I asked quite a lot of questions about Coq, and he kindly discussed with me every time, even though some questions looked a bit stupid.
Particularly on the \hemiola{} framework, Cl\'ement explained me how to reify shallowly embedded programs in a beautiful way.
Also, Stella and Thomas for discussing various hardware topics with me and more importantly for checking in on me regularly to see if I am doing fine.

I am thankful for belonging to another research group and also want to give my gratitude to the CSG/Arvind group members.
Very big gratitude to Chanwoo Chung, who has been my lunchmate for over 6 years, for teaching me everything about the real ``hardware'' -- from basic Bluespec knowledge to how to set up FPGAs on a server.
I would not have been able to synthesize any of my \hemiola{} cache-coherence protocols without his help.
I also thank the Riscy-processors developers -- Andy Wright, Sizhuo Zhang, and Thomas Bourgeat -- for kindly answering my questions about processor designs in Bluespec.
Thanks to Shuotao Xu as well for empathizing with the pain and pleasure of parenting.

A couple of people gave me invaluable intuitions for the \hemiola{} project.
I am very thankful to Professor Mengjia Yan for sharing the knowledge about noninclusive caches and state-of-the-art cache-coherence designs.
Also to Tej Chajed for discussing the relation between serializability and behavioral refinement, which was one of the main headaches building the framework.

There are some people who helped me a lot in terms of my mental health during my PhD journey.
The first four years of my PhD were with music; I was very lucky to receive the MIT Emerson scholarship for private music study and to perform various music performances.
I am very grateful to MIT Music \& Theater Arts (MTA) department to give me an opportunity to earn the scholarship.
I want to especially thank Professor Marilyn Roth at New England Conservatory (NEC) to teach me piano and precious ``life lessons'' for four years.
I also thank all the people who came to my first-and-last solo piano recital at MIT.

Last but not least, I would like to give my special gratitude to my family.
I would like to thank my parents, Kija Pyo and Seungil Choi, for having been very supportive for almost 26 years of my student life and having encouraged me to pursue academic careers.
My biggest gratitude goes to my wife Jeaeun Shon, for having been with me, overcoming difficulties in our lives, and empathizing with all my oppression, depression, and pressure during my big journey.
We have learned (mostly from COVID-19 happened from the year 2020) that life is not that easy and does not go as planned most of the times, but I am very sure we will find and go to the path that we are happy enough.
Lastly, I am very thankful to our lovely 2-month-old daughter Hannah Choi, for crying less than usual when I write these acknowledgements!

\emph{I dedicate this dissertation to my wife and daughter.}
