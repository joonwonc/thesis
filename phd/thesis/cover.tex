\title{Structural Design and Proof of\\Hierarchical Cache-Coherence Protocols}

\author{Joonwon Choi}
\prevdegrees{M.S., Massachusetts Institute of Technology (2016)\\B.S., Seoul National University (2013)}
\department{Department of Electrical Engineering and Computer Science}
\degree{Doctor of Philosophy in Electrical Engineering and Computer Science}

\degreemonth{February}
\degreeyear{2021}
\thesisdate{Jan 27, 2021}

%% By default, the thesis will be copyrighted to MIT.  If you need to copyright
%% the thesis to yourself, just specify the `vi' documentclass option.  If for
%% some reason you want to exactly specify the copyright notice text, you can
%% use the \copyrightnoticetext command.
%\copyrightnoticetext{\copyright IBM, 1990.  Do not open till Xmas.}

% If there is more than one supervisor, use the \supervisor command
% once for each.
\supervisor{Adam Chlipala}{Associate Professor of Electrical Engineering and Computer Science}

% This is the department committee chairman, not the thesis committee
% chairman.  You should replace this with your Department's Committee
% Chairman.
\chairman{Leslie A. Kolodziejski}{Professor of Electrical Engineering and Computer Science\\Chair, Department Committee on Graduate Students}

% Make the titlepage based on the above information.  If you need
% something special and can't use the standard form, you can specify
% the exact text of the titlepage yourself.  Put it in a titlepage
% environment and leave blank lines where you want vertical space.
% The spaces will be adjusted to fill the entire page.  The dotted
% lines for the signatures are made with the \signature command.
\maketitle

% The abstractpage environment sets up everything on the page except
% the text itself.  The title and other header material are put at the
% top of the page, and the supervisors are listed at the bottom.  A
% new page is begun both before and after.  Of course, an abstract may
% be more than one page itself.  If you need more control over the
% format of the page, you can use the abstract environment, which puts
% the word "Abstract" at the beginning and single spaces its text.

%% You can either \input (*not* \include) your abstract file, or you can put
%% the text of the abstract directly between the \begin{abstractpage} and
%% \end{abstractpage} commands.

% First copy: start a new page, and save the page number.
\cleardoublepage
% Uncomment the next line if you do NOT want a page number on your
% abstract and acknowledgments pages.
% \pagestyle{empty}
\setcounter{savepage}{\thepage}
\begin{abstractpage}
  Cache-coherence protocols have been one of the greatest correctness challenges of the hardware world.
  A memory subsystem usually consists of several caches and the main memory, and a cache-coherence protocol defined in such a system allows multiple memory-access transactions to execute in a distributed manner, across the levels of a cache hierarchy.
  This source of concurrency is the most challenging part in formal verification of cache coherence.

  In this dissertation, we introduce \hemiola{}, a framework embedded in Coq to design, prove, and synthesize cache-coherence protocols in a structural way.
  The framework guides the user to design protocols that never experience inconsistent interleavings while handling transactions concurrently.
  Any protocol designed in \hemiola{} always satisfies the \emph{serializability} property, allowing a user to prove the protocol assuming that \emph{transactions are executed one-at-a-time}.
  The proof relies on conditions on the protocol topology and state-change rules, but we have designed a domain-specific protocol language that guides the user to design protocols that satisfy these properties by construction.

  The framework also provides a novel way to design and prove invariants by adding predicates to messages in the system, called \emph{predicate messages}.
  On top of serializability, it is much simpler to prove a predicate message, since it is guaranteed that the predicate is not spuriously broken by other messages.

  We used \hemiola{} to design and prove hierarchical MSI and MESI protocols, in both inclusive and noninclusive variants, as case studies.
  We also demonstrated that the case-study protocols are indeed hardware-synthesizable, by using a compilation/synthesis toolchain in the framework.
\end{abstractpage}

% Additional copy: start a new page, and reset the page number.  This way,
% the second copy of the abstract is not counted as separate pages.
% Uncomment the next 6 lines if you need two copies of the abstract
% page.
% \setcounter{page}{\thesavepage}
% \begin{abstractpage}
% % $Log: abstract.tex,v $
% Revision 1.1  93/05/14  14:56:25  starflt
% Initial revision
% 
% Revision 1.1  90/05/04  10:41:01  lwvanels
% Initial revision
% 
%
%% The text of your abstract and nothing else (other than comments) goes here.
%% It will be single-spaced and the rest of the text that is supposed to go on
%% the abstract page will be generated by the abstractpage environment.  This
%% file should be \input (not \include 'd) from cover.tex.

Hardware components are extremely complex due to concurrency.
Modularity has been considered as an effective way to design and
understand such complex hardware components. Among various Hardware
Description Languages (HDLs), \Bluespec{} allows designers to develop
hardware not only based on modularity, but also based on the notion of
Guarded Atomic Actions (GAAs). Following the concepts of modularity
and GAA, we have been defining a framework called \Kami{}, which is
for specifying, verifying, and synthesizing \Bluespec{}-style hardware
components. However, modular semantics has an inherent weakness in
that it is hard to infer internal changes. In this thesis, I present a
new semantic approach based on inlining. Inlining semantics is defined
for open hardware systems and resolves the weakness by construction.
An implication from modular semantics to inlining semantics is also
formally proven, thus the inlining semantics can be used to
efficiently prove hardware properties.


% \end{abstractpage}

\cleardoublepage

\section*{Acknowledgments}

%% When I was in the first year of my PhD, I happened to watch a very popular comic strip (probably only among graduate students) named ``PHD Comics''; this particular comic strip\footnote{PHD Comics: Professed? \url{http://phdcomics.com/comics.php?f=1672}} was composed only of four panels, with captions ``Impressed! Oppressed. Depressed. and Mostly Pressed,'' summarizing a life of a graduate student.
%% I wondered if my student life would be like this cartoon.

%% After 6.5 years of my PhD, now I recall this comic strip in this dissertation and think about it again.
%% Indeed it was.
%% I was impressed by brilliant faculties and students who I worked with.
%% I was oppressed by research challenges that I had to get over.
%% I was depressed by a number of rejections I got from conferences and the unknown future.
%% I have felt constant pressure that I should make good research outcomes.

%% That being said, I have never thought it was a bad decision that I joined MIT as a graduate student.
%% I am rather grateful that I could do meaningful research in a great environment.
%% I am also grateful to myself for overcoming all the oppression, depression, and pressure that I have had during the PhD period.

%% Advisors

%% Thesis committee members

%% Colleagues in the PLV and CSG groups

%% Other people

%% Family

%% I would like to thank a number of people who have contributed to this
%% work or have given me thoughtful advice. Without help from these
%% people, I would not have been able to develop the work and to write
%% the thesis.

%% First, I would like to thank my advisors, Professor Arvind and Adam
%% Chlipala for giving me a significant amount of intuitions and
%% motivations for the work. Before joining the master degree program, I
%% had almost no experience in hardware verification. Thanks to advice
%% from Arvind and Adam, I could learn a lot of knowledge and skills fast
%% and efficiently.

%% I would also like to thank the CSG and PLV group people for invaluable
%% advice. I especially give my gratitude to Muralidaran Vijayaraghavan
%% for discussing and developing the \Kami{} project with me.

%% Last but not least, I would like to thank Kwanjeong Educational
%% Foundation, which supported me financially for my master's degree.
