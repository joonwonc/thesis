% -*- Mode:TeX -*-

%% IMPORTANT: The official thesis specifications are available at:
%%            http://libraries.mit.edu/archives/thesis-specs/
%%
%%            Please verify your thesis' formatting and copyright
%%            assignment before submission.  If you notice any
%%            discrepancies between these templates and the 
%%            MIT Libraries' specs, please let us know
%%            by e-mailing thesis@mit.edu

%% The documentclass options along with the pagestyle can be used to generate
%% a technical report, a draft copy, or a regular thesis.  You may need to
%% re-specify the pagestyle after you \include  cover.tex.  For more
%% information, see the first few lines of mitthesis.cls. 

%\documentclass[12pt,vi,twoside]{mitthesis}
%%
%%  If you want your thesis copyright to you instead of MIT, use the
%%  ``vi'' option, as above.
%%
%\documentclass[12pt,twoside,leftblank]{mitthesis}
%%
%% If you want blank pages before new chapters to be labelled ``This
%% Page Intentionally Left Blank'', use the ``leftblank'' option, as
%% above. 

\documentclass[12pt,twoside]{mitthesis}
\usepackage{lgrind}
%% These have been added at the request of the MIT Libraries, because
%% some PDF conversions mess up the ligatures.  -LB, 1/22/2014
\usepackage{cmap}
\usepackage[T1]{fontenc}
\pagestyle{plain}

%% This bit allows you to either specify only the files which you wish to
%% process, or `all' to process all files which you \include.
%% Krishna Sethuraman (1990).

\typein [\files]{Enter file names to process, (chap1,chap2 ...), or `all' to
process all files:}
\def\all{all}
\ifx\files\all \typeout{Including all files.} \else \typeout{Including only \files.} \includeonly{\files} \fi

\begin{document}

% -*-latex-*-
% 
% For questions, comments, concerns or complaints:
% thesis@mit.edu
% 
%
% $Log: cover.tex,v $
% Revision 1.8  2008/05/13 15:02:15  jdreed
% Degree month is June, not May.  Added note about prevdegrees.
% Arthur Smith's title updated
%
% Revision 1.7  2001/02/08 18:53:16  boojum
% changed some \newpages to \cleardoublepages
%
% Revision 1.6  1999/10/21 14:49:31  boojum
% changed comment referring to documentstyle
%
% Revision 1.5  1999/10/21 14:39:04  boojum
% *** empty log message ***
%
% Revision 1.4  1997/04/18  17:54:10  othomas
% added page numbers on abstract and cover, and made 1 abstract
% page the default rather than 2.  (anne hunter tells me this
% is the new institute standard.)
%
% Revision 1.4  1997/04/18  17:54:10  othomas
% added page numbers on abstract and cover, and made 1 abstract
% page the default rather than 2.  (anne hunter tells me this
% is the new institute standard.)
%
% Revision 1.3  93/05/17  17:06:29  starflt
% Added acknowledgements section (suggested by tompalka)
% 
% Revision 1.2  92/04/22  13:13:13  epeisach
% Fixes for 1991 course 6 requirements
% Phrase "and to grant others the right to do so" has been added to 
% permission clause
% Second copy of abstract is not counted as separate pages so numbering works
% out
% 
% Revision 1.1  92/04/22  13:08:20  epeisach

% NOTE:
% These templates make an effort to conform to the MIT Thesis specifications,
% however the specifications can change.  We recommend that you verify the
% layout of your title page with your thesis advisor and/or the MIT 
% Libraries before printing your final copy.
\title{An Inlining Approach to Formal Hardware Semantics}

\author{Joonwon Choi}
% If you wish to list your previous degrees on the cover page, use the 
% previous degrees command:
%       \prevdegrees{A.A., Harvard University (1985)}
% You can use the \\ command to list multiple previous degrees
%       \prevdegrees{B.S., University of California (1978) \\
%                    S.M., Massachusetts Institute of Technology (1981)}
\prevdegrees{B.S., Seoul National University (2013)}
\department{Department of Electrical Engineering and Computer Science}

% If the thesis is for two degrees simultaneously, list them both
% separated by \and like this:
% \degree{Doctor of Philosophy \and Master of Science}
\degree{Master of Science in Electrical Engineering and Computer Science}

% As of the 2007-08 academic year, valid degree months are September, 
% February, or June.  The default is June.
\degreemonth{June}
\degreeyear{2016}
\thesisdate{May 19, 2016}

%% By default, the thesis will be copyrighted to MIT.  If you need to copyright
%% the thesis to yourself, just specify the `vi' documentclass option.  If for
%% some reason you want to exactly specify the copyright notice text, you can
%% use the \copyrightnoticetext command.  
%\copyrightnoticetext{\copyright IBM, 1990.  Do not open till Xmas.}

% If there is more than one supervisor, use the \supervisor command
% once for each.
\supervisor{Arvind}
           {Professor of Electrical Engineering and Computer Science}

% This is the department committee chairman, not the thesis committee
% chairman.  You should replace this with your Department's Committee
% Chairman.

\chairman{Professor Leslie A. Kolodziejski}
         {Chair, Department Committee on Graduate Students}

% Make the titlepage based on the above information.  If you need
% something special and can't use the standard form, you can specify
% the exact text of the titlepage yourself.  Put it in a titlepage
% environment and leave blank lines where you want vertical space.
% The spaces will be adjusted to fill the entire page.  The dotted
% lines for the signatures are made with the \signature command.
\maketitle

% The abstractpage environment sets up everything on the page except
% the text itself.  The title and other header material are put at the
% top of the page, and the supervisors are listed at the bottom.  A
% new page is begun both before and after.  Of course, an abstract may
% be more than one page itself.  If you need more control over the
% format of the page, you can use the abstract environment, which puts
% the word "Abstract" at the beginning and single spaces its text.

%% You can either \input (*not* \include) your abstract file, or you can put
%% the text of the abstract directly between the \begin{abstractpage} and
%% \end{abstractpage} commands.

% First copy: start a new page, and save the page number.
\cleardoublepage
% Uncomment the next line if you do NOT want a page number on your
% abstract and acknowledgments pages.
% \pagestyle{empty}
\setcounter{savepage}{\thepage}
\begin{abstractpage}
% $Log: abstract.tex,v $
% Revision 1.1  93/05/14  14:56:25  starflt
% Initial revision
% 
% Revision 1.1  90/05/04  10:41:01  lwvanels
% Initial revision
% 
%
%% The text of your abstract and nothing else (other than comments) goes here.
%% It will be single-spaced and the rest of the text that is supposed to go on
%% the abstract page will be generated by the abstractpage environment.  This
%% file should be \input (not \include 'd) from cover.tex.

Hardware components are extremely complex due to concurrency.
Modularity has been considered as an effective way to design and
understand such complex hardware components. Among various Hardware
Description Languages (HDLs), \Bluespec{} allows designers to develop
hardware not only based on modularity, but also based on the notion of
Guarded Atomic Actions (GAAs). Following the concepts of modularity
and GAA, we have been defining a framework called \Kami{}, which is
for specifying, verifying, and synthesizing \Bluespec{}-style hardware
components. However, modular semantics has an inherent weakness in
that it is hard to infer internal changes. In this thesis, I present a
new semantic approach based on inlining. Inlining semantics is defined
for open hardware systems and resolves the weakness by construction.
An implication from modular semantics to inlining semantics is also
formally proven, thus the inlining semantics can be used to
efficiently prove hardware properties.


\end{abstractpage}

% Additional copy: start a new page, and reset the page number.  This way,
% the second copy of the abstract is not counted as separate pages.
% Uncomment the next 6 lines if you need two copies of the abstract
% page.
% \setcounter{page}{\thesavepage}
% \begin{abstractpage}
% % $Log: abstract.tex,v $
% Revision 1.1  93/05/14  14:56:25  starflt
% Initial revision
% 
% Revision 1.1  90/05/04  10:41:01  lwvanels
% Initial revision
% 
%
%% The text of your abstract and nothing else (other than comments) goes here.
%% It will be single-spaced and the rest of the text that is supposed to go on
%% the abstract page will be generated by the abstractpage environment.  This
%% file should be \input (not \include 'd) from cover.tex.

Hardware components are extremely complex due to concurrency.
Modularity has been considered as an effective way to design and
understand such complex hardware components. Among various Hardware
Description Languages (HDLs), \Bluespec{} allows designers to develop
hardware not only based on modularity, but also based on the notion of
Guarded Atomic Actions (GAAs). Following the concepts of modularity
and GAA, we have been defining a framework called \Kami{}, which is
for specifying, verifying, and synthesizing \Bluespec{}-style hardware
components. However, modular semantics has an inherent weakness in
that it is hard to infer internal changes. In this thesis, I present a
new semantic approach based on inlining. Inlining semantics is defined
for open hardware systems and resolves the weakness by construction.
An implication from modular semantics to inlining semantics is also
formally proven, thus the inlining semantics can be used to
efficiently prove hardware properties.


% \end{abstractpage}

\cleardoublepage

\section*{Acknowledgments}

I would like to thank a number of people who have been contributing to
this work or who gave me thoughtful advices. Without help from these
people, I would not have been able to develop the work and to write
the thesis.

First, I would like to thank my advisors, Professor Arvind and Adam
Chlipala for giving me a significant amount of intuitions and
motivations about the work. Before joining the master degree program,
I had almost no experience on hardware verification. Thanks for
advices from Arvind and Adam, I learned a lot of knowledge and skills
fast and efficiently.

I would also like to thank the CSG and PLV group people for invaluable
advices. I specially give my gratitude to Muralidaran Vijayaraghavan
for discussing and developing the \Kami{} project together.

Last but not least, I would like to thank Kwanjeong Educational
Foundation, who supported me financially for the master degree.

%%%%%%%%%%%%%%%%%%%%%%%%%%%%%%%%%%%%%%%%%%%%%%%%%%%%%%%%%%%%%%%%%%%%%%
% -*-latex-*-

% Some departments (e.g. 5) require an additional signature page.  See
% signature.tex for more information and uncomment the following line if
% applicable.
% % -*- Mode:TeX -*-
%
% Some departments (e.g. Chemistry) require an additional cover page
% with signatures of the thesis committee.  Please check with your
% thesis advisor or other appropriate person to determine if such a 
% page is required for your thesis.  
%
% If you choose not to use the "titlepage" environment, a \newpage
% commands, and several \vspace{\fill} commands may be necessary to
% achieve the required spacing.  The \signature command is defined in
% the "mitthesis" class
%
% The following sample appears courtesy of Ben Kaduk <kaduk@mit.edu> and
% was used in his June 2012 doctoral thesis in Chemistry. 

\begin{titlepage}
\begin{large}
This doctoral thesis has been examined by a Committee of the Department
of Chemistry as follows:

\signature{Professor Jianshu Cao}{Chairman, Thesis Committee \\
   Professor of Chemistry}

\signature{Professor Troy Van Voorhis}{Thesis Supervisor \\
   Associate Professor of Chemistry}

\signature{Professor Robert W. Field}{Member, Thesis Committee \\
   Haslam and Dewey Professor of Chemistry}
\end{large}
\end{titlepage}


\pagestyle{plain}
  % -*- Mode:TeX -*-
%% This file simply contains the commands that actually generate the table of
%% contents and lists of figures and tables.  You can omit any or all of
%% these files by simply taking out the appropriate command.  For more
%% information on these files, see appendix C.3.3 of the LaTeX manual. 
\tableofcontents
\newpage
\listoffigures
%% \newpage
%% \listoftables


\chapter{Protocol Transition Systems and Serializability}

\section{Protocol Transition Systems}
\label{sec-trs}

\paragraph{Notations.}
Throughout the thesis, we will use several notations for lists (sequences) and finite maps.
An overline (\eg{} $\listof{l}$) denotes a list.
$\llistof{l}$ denotes a list of lists.
$\listnil{}$, $(\listcons{\listof{l}}{e})$, $(\listapp{\listof{l_1}}{\listof{l_2}})$, $(\listsub{\listof{l_1}}{\listof{l_2}})$, and
$(\listdisj{\listof{l_1}}{\listof{l_2}})$ denote nil, single-element append, general append, subtraction, and disjointness of lists, respectively.
We use the same operation $(+)$ for the single-element and general append.
$\listconcat{\llistof{l}}$ flattens the list of lists $\llistof{l}$ with repeated concatenation.
$\sizeof{\listof{l}}$ is the length of a list.

Regarding a list of key-value pairs as a finite map, we override notations for lists.
For example, $(\mapupds{M}{\listof{l}})$ updates multiple key-value pairs in a finite map $M$.
Moreover, we overload the same operation $(\mapupd{M}{k}{v})$ for a single update for simplicity.

We will use $\tuple{\cdot}$ to denote a struct and use a name (\eg{} $s.\textsf{fd}$) to access a field value.
$(\listof{s.\textsf{fd}})$ will be used as a shorter notation for $(\textsf{List.map}\ (\lambda s.\; s.\textsf{fd})\ \listof{s})$.

\subsection{Syntax}
\label{sec-syntax}

\begin{figure}[t]
  \centering
  \begin{tabular}{|c|}
    \hline
    \begin{math}
      \begin{array}{rl}
        \textrm{ID} & \msgid{} \in \hidxt{} \\
        \textrm{Value} & \msgval{} \in \hvaluet{} \\
        \textrm{Message} & m ::= \msgbuild{\msgty}{\msgid}{\msgval} \in \hmsgt{} \triangleq \boolt \ast \hidxt \ast \hvaluet \\
        \textrm{Index} & i \in \hidxt{}\ \textrm{(for channels, objects, etc.)} \\
        \textrm{Channel Index \& Message} & im ::= \idmbuild{i}{m} \in \hidmt \triangleq \hidxt \ast \hmsgt \\
        \textrm{Object state} & o \in \hostt{} \\
        \textrm{Rule precondition} & \ruleprec{} \in \hostt \times \listtof{\hidmt} \to \propt \\
        \textrm{Rule transition} & \ruletrs{} \in \hostt \times \listtof{\hidmt} \to \hostt \times \listtof{\hidmt}\\
      \end{array}
    \end{math}\\
    \hline
    \begin{math}
      \begin{array}{rl}
        \textrm{Rule} & r ::= \tuple{i, \ruleprec{}, \ruletrs{}} \\
        \textrm{Object} & O ::= \tuple{i, \objInit{o}, \listof{r}} \\
        \textrm{System} & S ::= \hsyss{\listof{O}}{i} \\
      \end{array}
    \end{math}\\
    \hline
  \end{tabular}
  \caption{Protocol transition system}
  \label{fig-trs-system}
\end{figure}

\hemiola{} uses formal protocol transition systems as an underlying basis for reasoning about cache-coherence protocols.
The protocol transition systems faithfully formalize conventional message-passing systems, but they are more restrictive to exclude some behaviors that cannot happen in hardware.
We will see such restrictions in detail while explaining the semantics in \autoref{sec-semantics}.

\autoref{fig-trs-system} explains what such systems are.
A \emph{message} $m$ is a communication unit, consisting of a Boolean message type, a message ID, and a value.
A message type is false (true) for a request (response), respectively.
A message ID is an enumeration of message kinds.
We use \emph{value} to refer to each line in a cache or a memory.
Note that a struct sometimes has an \emph{index} to distinguish it from the other components.
A pair \idmbuild{i}{m} is used to represent a message $m$ residing in a channel with an index $i$.

Rules make local state transitions within an object.
A rule $r$ is a struct composed of its rule index, a precondition (\ruleprec{}), and a transition function (\ruletrs{}).
Each rule has a unique index within an object.
A precondition \ruleprec{} takes two arguments, a current object state and input messages (as a list of pairs \idmbuild{i}{m}), and decides whether the rule can be executed or not with the current state.
A transition function takes the same arguments but returns the next object state and output messages (also as a list of \idmbuild{i}{m}).
Note that our formalization of the protocol transition system is shallowly embedded in Coq, \eg{} precondition and transition definitions use native Coq function types.

An object $O$ contains its object index (unique within a system), an initial state ($\objInit{o}$), and rules (\listof{r}) that make local state transitions within the object.
The highest-level component is a system $S$, which contains information about objects and channels.
It consists of objects (\listof{O}) and channel indices for internal messages (\hsysIn{i}), external inputs (\hsysRq{i}), and external outputs (\hsysRs{i}).
The definition is general in that any object can access any channel in the system, just by mentioning the channel index in a state transition.
It is necessary to distinguish between internal and external channels, in order to define external behaviors of the system, \ie{} the interface of the cache-coherence protocol with processor cores, which will be explained in \autoref{sec-semantics}.

\subsection{Semantics}
\label{sec-semantics}

\subsubsection{State-Transition Steps}

\begin{figure}[t]
  \centering
  \begin{tabular}{|c|}
    \hline
    \multicolumn{1}{|l|}{\textbf{Types:}} \\
    \begin{tabular}{lr}
      $\textrm{Message States}\ \ M \in \hidxt{} \to \listtof{\hmsgt}$ &
      $\textrm{State}\ \ s \in \hstt{} ::= \tuple{\listof{o}, M}$ \\
      \multicolumn{2}{l}{$\textrm{Label}\ \ l ::= \lblEmpty{}\; |\; \lblIns{\listof{im}}\; |\; \lblOuts{\listof{im}}\; |\; \lblInt{\idxOf{O}}{\idxOf{r}}{\listof{im}}{\listof{im}}$} \\
    \end{tabular}\\
    \multicolumn{1}{|l|}{\textbf{Step:}} \\
    \begin{math}
      \begin{array}{c}
        \inference[StepSilent:]{}{\semstep{S}{s}{\lblEmpty{}}{s}}\bigskip\\
        \inference[StepIns:]{\listof{im} \neq \listnil
          & \listof{\idxOf{im}} \subseteq \hsysRqA{S}}{\semstep{S}
          {\hst{\listof{o}}{M}}
          {\lblIns{\listof{im}}}
          {\hst{\listof{o}}{\enqMsgs{M}{\listof{im}}}}}\bigskip \\
        \inference[StepOuts:]{\listof{im} \neq \listnil
          & \listof{im} \subseteq \heads{M}
          & \listof{\idxOf{im}} \subseteq \hsysRsA{S}}{\semstep{S}
          {\hst{\listof{o}}{M}}
          {\lblOuts{\listof{im}}}
          {\hst{\listof{o}}{\deqMsgs{M}{\listof{im}}}}}\bigskip \\
        \inference[StepInt:]{S = \hsyss{\listof{O}}{i}
          & O \in S.\listof{O}
          & r \in O.\listof{r}\smallskip \\
          \midxIns{im} \subseteq \hsysInA{S} \cup \hsysRqA{S}
          & \listof{o}[\idxOf{O}] = o_1
          & \msgIns{im} \subseteq \heads{M}\smallskip \\
          r.p\enspace o_1\ \msgIns{im}
          & r.t\ o_1\ \msgIns{im} = (o_2, \msgOuts{im})\smallskip \\
          \midxOuts{im} \subseteq \hsysInA{S} \cup \hsysRsA{S}
          & \disj{\midxIns{im}}{\midxOuts{im}}}{\semstep{S}
          {\hst{\listof{o}}{M}}
          {\lblInt{\idxOf{O}}{\idxOf{r}}{\msgIns{im}}{\msgOuts{im}}}
          {\hstm{\mapupd{\listof{o}}{\idxOf{O}}{o_2}}{\enqMsgs{\deqMsgs{M}{\msgIns{im}}}{\msgOuts{im}}}}}\medskip \\
      \end{array}
    \end{math}\\
    \hline
  \end{tabular}
  \caption{Step semantics in protocol transition systems}
  \label{fig-trs-semantics-steps}
\end{figure}

\autoref{fig-trs-semantics-steps} describes the semantics for state-transition steps in a protocol transition system.
A state transition (step) happens by a rule that takes input messages, makes an object-state transition, and generates output messages.
The semantics for a step is presented as a judgment \semstep{S}{s_0}{l}{s_1}, where $S$ is the system to execute, $s_0$ is a prestate, $s_1$ is a poststate, and $l$ is a label generated by the state transition.
The state of a system (in domain $\hstt{}$) is a pair of object states and message states.
Object states are represented in a finite map from object indices to object states.
Message states are also represented in a finite map from channel indices to ordered queues of messages.

From now on, we assume that all the input and output messages used in the step definitions do not share the same channel, \ie{} $(\nodup{\listof{\idxOf{im}}})$.
In other words, while taking inputs and generating outputs, each step case never accesses a channel twice.
It may be possible (and practical) in conventional message-passing systems to dequeue multiple messages from the same queue or to enqueue multiple ones to the same queue.
In hardware, however, this is never a practical usage of ordered channels (FIFOs).
Particularly, in cache-coherence protocols, we cannot imagine any case accessing the same FIFO at least twice in the same clock cycle.

Rule [StepSilent] represents the case where no state transition happens in the current step; an empty label (\lblEmpty{}) is generated in this case.
A system may accept input messages from the external world.
[StepIns] describes this case, where the external input messages (\listof{im}) should not be empty ($\listof{im} \neq \listnil$), and channels of the messages are valid ($\listof{\idxOf{im}} \subseteq \hsysRqA{S}$), \ie{} the input messages are all put to external-request channels.
An external-inputs label $(\lblIns{\listof{im}})$ is generated in this case.
[StepOuts] describes the opposite case, for output messages being released to the external world.
In this case, in addition to the [StepIns] case, each output message should be in the head (the first element) of its residing channel ($\listof{im} \subseteq \heads{M}$).

Lastly, [StepInt] deals with a state transition by a rule ($r$) in an object ($O$).
It nondeterministically chooses an object and a rule in the object, checks that the precondition holds ($r.p\enspace (o_1, \msgIns{im})$), and applies the transition to update the state of the system ($r.t\enspace (o_1, \msgIns{im}) = (o_2, \msgOuts{im})$).
An internal label ($\lblInt{\idxOf{O}}{\idxOf{r}}{\msgIns{im}}{\msgOuts{im}}$) is generated in this case, which records an object index, a rule index, input messages, and output messages.
Each input message should be from either an internal channel or an external-request one ($\midxIns{im} \subseteq \hsysInA{S} \cup \hsysRqA{S}$) and should be the first element of the channel ($\msgIns{im} \subseteq \heads{M}$).
On contrary to the input messages, each output message should be enqueued to either an internal channel or an external-response one ($\midxOuts{im} \subseteq \hsysInA{S} \cup \hsysRsA{S}$).
Lastly, the channels of the input and output messages should be disjoint to each other ($\disj{\midxIns{im}}{\midxOuts{im}}$).
Note that the semantics is based on ordered channels, so messages are \emph{enqueued} and \emph{dequeued} in each state-transition case.
We use notations $\enqMsgs{M}{\listof{im}}$ and $\deqMsgs{M}{\listof{im}}$ for such operations.

%%       \multicolumn{1}{l}{\textbf{(c) Atomic histories:}} \\
%%       \begin{math}
%%         \begin{array}{c}
%%           \inference[AtomicStart:]{}{\atomicLong{\listof{\amsgi{im}}}{\listsingle{\lblInt{\idxOf{O}}{\idxOf{r}}{\listof{\amsgi{im}}}{\listof{\amsge{im}}}}}{28}{\listof{\amsge{im}}}}\\[15pt]
%%           \inference[AtomicCont:]{\atomic{\listof{\amsgi{im}}}{\listof{l}}{\listof{\amsge{im}}}
%%             & \msgIns{n} \neq \listnil
%%             & \msgIns{n} \subseteq \listof{\amsge{im}}}{\atomicLong{\listof{\amsgi{im}}}{\listcons{\listof{l}}{\lblInt{\idxOf{O}}{\idxOf{r}}{\msgIns{n}}{\msgOuts{n}}}}{28}{(\listapp{\listsub{\listof{\amsge{im}}}{\msgIns{n}}}{\msgOuts{n}}})} \\
%%           \mbox{}\vspace{-5pt} %% padding
%%         \end{array}
%%       \end{math}\\
%%       \hline
%%       \multicolumn{1}{l}{\textbf{(d) Transactions:}} \\
%%       \begin{math}
%%         \arraycolsep=5pt
%%         \begin{array}{cc}
%%           \begin{array}{c}
%%             \inference[TrsSilent:]{}{\trsn{S}{\listsingle{\lblEmpty{}}}} \\[5pt]
%%             \inference[TrsIns:]{}{\trsn{S}{\listsingle{\lblIns{\listof{im}}}}} \\[5pt]
%%             \inference[TrsOuts:]{}{\trsn{S}{\listsingle{\lblOuts{\listof{im}}}}} \\[5pt]
%%             \mbox{}\vspace{-5pt} %% padding
%%           \end{array} &
%%           \begin{array}{l}
%%             \textrm{\footnotesize TrsAtomic:} \\
%%             \inference[]{\extatomic{S}{\listof{\amsgi{im}}}{\listof{l}}{\listof{\amsge{im}}}}{\trsn{S}{\listof{l}}}
%%           \end{array} \\
%%         \end{array}
%%       \end{math}\\
%%     \end{tabular}\\
%%     \hline
%%   \end{tabular}
%%   \caption{Steps, behaviors, atomic histories, and transactions in protocol transition systems}
%%   \label{fig-trs-semantics}
%% \end{figure}

\begin{figure}[t]
  \centering
  \begin{tabular}{|c|}
    \hline
    \multicolumn{1}{|l|}{\textbf{Steps and behaviors:}} \\
    \begin{math}
      \arraycolsep=10pt
      \begin{array}{cc}
        \inference[StepsNil:]{}{\semsteps{S}{s}{\listnil{}}{s}}
        & \inference[StepsCons:]{\semsteps{S}{s_0}{\listof{l}}{s_1}
          & \semstep{S}{s_1}{l_1}{s_2}}{\semsteps{S}{s_0}{\listcons{\listof{l}}{l_1}}{s_2}}\medskip \\
      \end{array}
    \end{math}\\
    \begin{tabular}{c}
      \inference[Behavior:]{\semsteps{S}{\sysInit{S}}{\listof{l}}{s}}{\sembeh{S}{\behOf{\listof{l}}}}\medskip\\
    \end{tabular}\\
    \hline
  \end{tabular}
  \caption{Multiple transition steps and behaviors in protocol transition systems}
  \label{fig-trs-semantics-steps-beh}
\end{figure}

The step semantics is naturally lifted to one for multiple steps, as shown in \autoref{fig-trs-semantics-steps-beh}.
It is presented as a judgment $\semsteps{S}{s_0}{\listof{l}}{s_1}$, where $S$ is the system to execute, $s_0$ is a prestate, $s_1$ is a poststate, and $\listof{l}$ is a \emph{sequence of labels} generated by executions of the steps.
[StepsNil] serves the case where no state transitions happen, and no labels are generated in this case.
[StepsCons] is a natural inductive constructor that combines previous steps (\semsteps{S}{s_0}{\listof{l}}{s_1}) and a new one (\semstep{S}{s_1}{l_1}{s_2}).
The label by the new step is appended to the last of the label sequence of the previous steps.

Throughout the thesis, we will now call a sequence of labels a \emph{history}.
We say that a state $s$ is \emph{reachable} iff there is a history $\listof{l}$ such that $\semsteps{S}{\sysInit{S}}{\listof{l}}{s}$ holds, where $\sysInit{S}$ is the initial state of the system $S$, constructed by composing all initial object states.
We use a simpler notation $\semrch{S}{s}$ for reachable states.
We also say that a history $\listof{l}$ is \emph{legal} iff there is a state $s$ such that $\semsteps{S}{\sysInit{S}}{\listof{l}}{s}$ holds.
We write $\semleg{S}{\listof{l}}$ to assert that a history is legal.

\subsubsection{Remarks on the transition steps in hardware}

\todo{Move to the Background section, maybe?}

%% It is worth discussing whether the transition-step definitions defined so far are appropriate for describing behaviors in hardware.
%% While only a single rule is executed and makes a state transition,

It is worth noting that we take for granted that rules are executed atomically, even when multiple rules are executed at the same time.
Here we appeal to a tradition of hardware-description approaches that guarantee rule-level serialization (also known as ``one-rule-at-a-time semantics'') for this kind of design~\cite{fesi,kami,Murali:2015,Dave:2005,Dave:2007}.

\subsubsection{Behaviors and correctness}

A system $S$ has a behavior $\behOf{\listof{l}}$, denoted as \sembeh{S}{\behOf{\listof{l}}}, if there exists an execution of steps that generates $\listof{l}$, starting with the initial state of $S$ ([Behavior] in \autoref{fig-trs-semantics-steps-beh}).
Here the $\behOf{\cdot}$ operation filters out silent ($l_\epsilon$) and internal ($l_{\textrm{int}}$) labels so only the external parts remain.
We call such a sequence of labels a \emph{trace}.
In other words, a trace only consists of external-inputs and external-outputs labels.

Finally we define trace refinement as a notion of correctness in protocol transition systems:
\begin{definition}[Trace Refinement]
  A system $I$ (``implementation'') trace-refines another system $S$ (``specification''), written as $\refines{I}{S}$, iff every trace of $I$ is also a trace of $S$:
  \begin{displaymath}
    \refines{I}{S} \triangleq \forall \listof{t}.\; \sembeh{I}{\listof{t}} \to \sembeh{S}{\listof{t}}.
  \end{displaymath}
\end{definition}

Trace refinement is one of the well-known correctness criteria to claim that the external (observable) behavior of a given implementation is within the behavior boundary of the specification.
In other words, by proving trace refinement, we can say that the implementation does not go wrong in terms of the specification.

How do we prove trace refinement for a given implementation and a spec?
It is usually proven by establishing a \emph{simulation} relation between the implementation and the spec states:
\begin{definition}[Simulation]
  We call $(\sim): \hstt{} \times \hstt{} \to \propt{}$ a simulation between the systems $I$ and $S$ iff 1) the relation holds for the initial states and 2) a step in $S$ exists for every step in $I$, which generates the same external label and preserves the relation:
  \begin{displaymath}
    \begin{array}{rl}
      1) & \sysInit{I} \sim \sysInit{S},\\
      2) & \forall s_0, s_1, l.\; \semstep{I}{s_0}{l}{s_1} \to \forall t_0.\; s_0 \sim t_0 \to \exists t_1.\; \semstep{S}{t_0}{l}{t_1} \wedge s_1 \sim t_1.
    \end{array}
  \end{displaymath}
\end{definition}

It is also well-known that simulation directly implies trace refinement~\cite{equivalence}, which is proven simply by induction on state-transition steps:
\begin{theorem}[Simulation implies trace refinement]
  If there is a simulation $(\sim)$ between two systems $I$ and $S$, then \refines{I}{S}.
\end{theorem}

\include{chap2}
\appendix
\chapter{Tables}

\begin{table}
\caption{Armadillos}
\label{arm:table}
\begin{center}
\begin{tabular}{||l|l||}\hline
Armadillos & are \\\hline
our	   & friends \\\hline
\end{tabular}
\end{center}
\end{table}

\clearpage
\newpage

\chapter{Figures}

\vspace*{-3in}

\begin{figure}
\vspace{2.4in}
\caption{Armadillo slaying lawyer.}
\label{arm:fig1}
\end{figure}
\clearpage
\newpage

\begin{figure}
\vspace{2.4in}
\caption{Armadillo eradicating national debt.}
\label{arm:fig2}
\end{figure}
\clearpage
\newpage

%% This defines the bibliography file (main.bib) and the bibliography style.
%% If you want to create a bibliography file by hand, change the contents of
%% this file to a `thebibliography' environment.  For more information 
%% see section 4.3 of the LaTeX manual.
\begin{singlespace}
\bibliography{main}
\bibliographystyle{plain}
\end{singlespace}

\end{document}

